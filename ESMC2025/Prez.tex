%% Options. 169 is now officially the format
%% aspectratio=169 : 169 is now officially the format
%% t : t is needed to get proper display.
%% 10pt: general font size 
%% rounded : use rounded box instead of rectangle
\documentclass[t,10pt,aspectratio=169]{beamer}


%% Load theme and commands
%% Possible options so far are subsidiary=XXX, XXX being either list-sac, list-gre, list-all
%% The subidiary option allows to inlcude few more logos
%% \usetheme[subsidiary=list-sac]{CEA23}
\usetheme{CEA23}

% encoding and language (choose english of french)
\usepackage[utf8]{inputenc}
\usepackage[french]{babel}
\usepackage[T1]{fontenc}
\usepackage{multimedia}
\usepackage{csquotes}
\usepackage[style=authoryear,backend=bibtex]{biblatex}

% packages
\usepackage[absolute,overlay]{textpos}
\usepackage{subcaption}
\usepackage{setspace}
\usepackage{amsmath,amsthm,mathrsfs}
\usepackage{amssymb}
\usepackage{array,multirow,makecell}
\newcolumntype{L}[1]{>{\arraybackslash }b{#1}}
\newcolumntype{C}[1]{>{\centering\arraybackslash }b{#1}}
\newcommand{\dbldot}{\mathbin{\mathord{:}}}
\renewcommand{\vec}[1]{\boldsymbol{\mathrm{#1}}}
\newcommand{\mat}[1]{\boldsymbol{\mathrm{#1}}}
\newcommand{\rot}{\textbf{\textit{\textrm{rot}}}\ }
\newcommand{\divv}{\textbf{\textit{\textrm{div}}}\ }
\newcommand{\D}{\mathrm{d}}
\renewcommand{\div}{\textit{\textrm{div}}\ }
\newcommand{\grad}{\mathrm{\nabla}}
\newcommand{\grads}{\mathrm{\nabla}^s}
\newcommand{\grada}{\mathrm{\nabla}^a}
\newcommand{\tens}[1]{\vec{#1}}
\newcommand{\e}[1]{\textit{\textrm{#1}}}
\newcommand{\ee}[1]{\textrm{#1}}
\newcommand{\vece}[1]{\textbf{\textit{\textrm{#1}}}}
\newcommand{\vecee}[1]{\textbf{\textrm{#1}}}
\newcommand{\eff}{{\textrm{eff}}}
\newcommand{\per}{{\textrm{per}}}
\newcommand{\subitem}[1]{\ \ \ \tiny{$\blacksquare$}\small{#1}}
\newcommand{\Eb}{\overline{\vec{E}}}
\newcommand{\beq}{\begin{equation}}
\newcommand{\eeq}{\end{equation}}
\newcommand{\beqal}{\begin{equation}\begin{aligned}}
\newcommand{\eeqal}{\end{aligned}\end{equation}}
\newcommand{\dtepsilon}{\dot{\tens\varepsilon}}
\newcommand{\dtsigma}{\dot{\tens\sigma}}
\newcommand{\Dtepsilon}{\Delta\tens\varepsilon}
\newcommand{\Dtsigma}{\Delta\tens\sigma}
\newcommand{\tepsilon}{\tens\varepsilon}
\newcommand{\tsigma}{\tens\sigma}

\bibliography{Prez_SESC}

\begin{document}

% classical beamer stuff
%% \title{This is our \LaTeX-based presentation title }
\title{Implementations of homogenized behaviours in structural codes: examples and on-going efforts on extending the \texttt{MFront} code generator}
\subtitle{~\\European Solid Mechanics Conference 2025}
\date[07/2025]{July 2025}
\author{Antoine \textsc{Martin}}

%% logo on title page
\titlepagelogo{\includegraphics[height=24mm]{graphics/ORIGINAL/LOGO_CEA.png}}

%% the possibilities are the original logo (LOGO_CEA.png) or the derived versions {AIF.png IRAMIS.png IRESNE.png IRFM.png
%% IRFU.png IRIG.png ISAS.png ISEC.png I-TESE.png JACOB.png JOLIOT.png LETI.png LIST.png LITEN.png}

%% It is also possible to have several logos !!!! Forbidden for public presentation (out of CEA) except for List specific
%% case seen above
%% \titlepagelogo{\includegraphics[height=24mm]{graphics/ORIGINAL/JOLIOT.png}\hspace{10pt}\includegraphics[height=24mm]{ATLAS_logo_default_RGBHEX.pdf}}

%% Define a diffusion level, usually empty but can be restreinte, restreintesf, S, TS
\setvalue{\diffusion}{} % restreinte, restreintesf, S, TS

% specific CEA beamer stuff
%\seminar[]{}


\CEAcenter{CEA CADARACHE}
\addr{Centre CEA Cadarache 13108 Saint Paul Lez Durance Cedex\\France}
\CEAemail{\href{mailto:antoine.martin@cea.fr}{\texttt{antoine.martin@cea.fr}}}

\begin{frame}[plain]
  \titlepage
\end{frame}
	\begin{frame}
	  \frametitle{The framework}
	  \begin{itemize}
		\item{A maxwellian linear viscoelastic case\\
		~\\}
		\item{A matrix-inclusion microstructure\\
		~\\}
		\item{We target the computation at the structure scale\\
		~\\}
	\end{itemize}
		
	
	 \end{frame}
	 
	 \begin{frame}
	  \frametitle{Preliminaries}
	  	\beqal
		&\quad\tens \alpha(\vec x)=\tens \alpha_c(\vec x)+\tens \alpha_s(\vec x)
		\eeqal
		with
		\beqal
		&\quad\tens \alpha_c(\vec x)=\overline{\tens \alpha}+\Gamma^0\left(\tens L_0\dbldot\tens \alpha\right)(\vec x)\\
		&\quad\tens \alpha_s(\vec x)=\tens \alpha(\vec x)-\overline{\tens \alpha}-\Gamma^0\left(\tens L_0\dbldot\tens \alpha\right)(\vec x)
		\eeqal
	 \end{frame}
	
	\begin{frame}
	  \frametitle{Non-linear visco-elasticity}
		\beqal
		t=0 :& \quad\tsigma(\vec x,0)=\tsigma_0(\vec x),\quad \tepsilon(\vec x,0)=\tepsilon_0(\vec x)\\
		t\geq0 :&\quad \tepsilon(\vec x,t)=\overline{\vec E}(t)+\grad^s\vec \xi^\per(\vec x,t)\\
		&\quad\dtsigma(\vec x,t)=\tens C(\vec x)\dbldot\left(\dtepsilon(\vec x,t)-f(\tsigma(\vec x,t))\right)\\
		&\quad\div\tsigma(\vec x,t)=0
		\eeqal
		We follow an implicit resolution:
		\beqal
		&\quad \Delta\tepsilon(\vec x)=\Delta\overline{\vec E}+\grad^s\Delta\vec \xi^\per(\vec x)\\
		&\quad\Delta\tsigma(\vec x)=\tens C(\vec x)\dbldot\left(\Delta\tepsilon(\vec x)-\Delta t\,f(\tsigma(\vec x,t)+\theta\Delta\tsigma)\right)\\
		&\quad\div\Delta\tsigma(\vec x)=0
		\eeqal
	 \end{frame}
	 
	 
	 \begin{frame}
	  \frametitle{The maxwellian visco-elasticity}
		\beqal
		&\quad\Delta\tsigma(\vec x)=\tens C(\vec x)\dbldot\left(\Delta\tepsilon(\vec x)-\Delta t\,\tens M(\vec x)\dbldot(\tsigma(\vec x,t)+\theta\Delta\tsigma)\right)\\
		&\quad\left(\tens I+\theta\Delta t\,\tens C(\vec x)\dbldot\tens M(\vec x)\right)\dbldot\Delta\tsigma(\vec x)=\tens C(\vec x)\dbldot\left(\Delta\tepsilon(\vec x)-\tens \alpha(\vec x)\right)\\
		\eeqal
		with:
		\beqal
		&\quad\tens \alpha(\vec x)=\Delta t\,\tens M(\vec x)\dbldot\tsigma(\vec x,t)
		\eeqal
		which is
		\beqal
		&\quad\Delta\tsigma(\vec x)=\tens L(\vec x)\dbldot\left(\Delta\tepsilon(\vec x)-\tens \alpha(\vec x)\right)\\
		\eeqal
		with
		\beqal
		&\quad \tens L(\vec x)=\left(\tens I+\theta\Delta t\,\tens C(\vec x)\dbldot\tens M(\vec x)\right)^{-1}\dbldot\tens C(\vec x)
		\eeqal
	 \end{frame}
	 
	 
	 
	 \begin{frame}
	  \frametitle{One phase is visco-elastic, the others are purely elastic}
		
	 \end{frame}
	 
	 
	 \begin{frame}
	  \frametitle{One phase is visco-elastic, the others are purely elastic}
		
	 \end{frame}
	 
	 
	 
	 
	 
	 \begin{frame}
	  \frametitle{Sum up and implementation}
	  	
		
	 \end{frame}

  
\end{document}








