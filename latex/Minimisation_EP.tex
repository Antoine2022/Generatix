\documentclass[times, final, 3p]{article}
\newcommand{\mytitle}{}
\newcommand{\myjournal}{}
\newcommand{\myauthors}{A. Martin}

\title{\mytitle}
%\journal{\myjournal}


\usepackage[fleqn]{amsmath}

\usepackage{amssymb}
\usepackage{amsthm}
\newtheorem{remark}{Remark}

\usepackage{bm}
\usepackage{tikz}

%\usepackage[breaklinks=true, colorlinks=true, linktocpage=true, pdftitle={\mytitle}, pdfauthor={\myauthors}, pdfsubject={Preprint submitted to \myjournal}, urlcolor=blue]{hyperref}


\usepackage{xcolor}

\newcommand{\app}{\text{app}}
\newcommand{\D}{\mathrm d}
\newcommand{\dbldot}{\mathbin{\mathord{:}}}
\let\div\undefined
\DeclareMathOperator{\div}{div}
\newcommand{\eff}{\mathrm{eff}}
\newcommand{\beqal}{\begin{equation}\begin{aligned}}
\newcommand{\eeqal}{\end{aligned}\end{equation}}
\DeclareMathOperator{\grad}{\mathbf{grad}}
\newcommand{\I}{\mathrm i}
\newcommand{\integers}{\mathbb Z}
\newcommand{\jump}[1]{[\![#1]\!]}
\newcommand{\M}{\mathrm{m}}
\newcommand{\per}{\mathrm{per}}
\newcommand{\PV}{\operatorname*{PV}}
\newcommand{\reals}{\mathbb R}
\DeclareMathOperator{\sym}{\mathbf{sym}}
\newcommand{\tens}[1]{\vec{#1}}
\renewcommand{\vec}[1]{\bm{\mathrm{#1}}}
\newcommand{\rot}[1]{\vec{rot}\ \vec{#1}}
\newcommand{\taumoy}{\overline{\vec{\tau}}}
\newcommand{\Eb}{\overline{\vec{E}}}

\bibliographystyle{elsarticle-harv}

\begin{document}


\section{Minimisation of potential energy}

We consider
a periodic cell \(Z\) of a random heterogeneous
material. The conductivity at \(\vec x\in Z\) is \(\tens\sigma(\vec x)\)
(symmetric, positive definite, second-order tensor); \(\vec E(\vec x)\),
\(\phi(\vec x)\) and \(\vec j(\vec x)\) denote the electric field, the electric
potential and the volumic current, respectively,
at point \(\vec x\). The apparent conductivity of the cell \(Z\),
\(\tens\sigma^\app\), is found from the solution to
the following problem
\begin{gather}
  \label{eq:20211209153049}
  Z:\quad\div{\vec{j}}=0,\\
  \label{eq:20211209153053}
  Z:\quad\vec{j}=\tens{\sigma}\cdot\vec{E},\\
  \label{eq:20211209153103}
  Z:\quad\vec{E}=\overline{\vec E}+\grad\phi^\per,
\end{gather}
where $\phi^\per$ is a $Z$-periodic field, $\vec{j}$ is a $Z$-antiperiodic field, and $\overline{\vec E}$
is a prescribed constant vector. Eq.~\eqref{eq:20211209153103} ensures that the electric field is curl-free,
with $\overline{\vec E}=\langle\vec E\rangle$, where angle brackets
denote volume averages over the cell \(Z\).
We consider a biphasic cell with a matrix whose conductivity is $\tens \sigma_0$ and another phase whose
conductivity is $\tens \sigma_1$.

We introduce the linear Green operator \(\tens{\Gamma}_0^\per\),
associated with the conductivity \(\tens\sigma_0\).

The electrical field solution of problem \eqref{eq:20211209153049}-\eqref{eq:20211209153103} minimizes the following energy:
\begin{equation}
\label{eq_EP}
  \mathcal{W}(\vec E)=\frac12\int_Z\vec E\cdot\tens\sigma\cdot\vec E.
\end{equation}

We then consider the following electrical field:
\begin{equation}
\vec E=\overline{\vec{E}}-\tens{\Gamma}_0^\per(\chi\overline{\vec{\tau}})
\end{equation}
where \(\overline{\vec \tau}\) is a constant vector and $\chi$ denotes characteristic function of medium whose conductivity is $\tens \sigma_1$.
This field is curl-free, and its average is $\overline{\vec{E}}$.
We then define
\begin{equation}
\begin{aligned}
\mathcal{E}(\overline{\vec{\tau}})&=\mathcal{W}\left(\overline{\vec{E}}-\tens{\Gamma}_0^\per(\chi\overline{\vec{\tau}})\right)\\
&=\frac12\int_Z\left(\overline{\vec{E}}-\tens{\Gamma}_0^\per(\chi\overline{\vec{\tau}})\right)\cdot\tens\sigma\cdot\left(\overline{\vec{E}}-\tens{\Gamma}_0^\per(\chi\overline{\vec{\tau}})\right)\\
&=\frac12\int_Z\left(\overline{\vec{E}}-\tens H\cdot\overline{\vec{\tau}}\right)\cdot\tens\sigma\cdot\left(\overline{\vec{E}}-\tens H\cdot\overline{\vec{\tau}}\right)
\end{aligned}
\end{equation}
where
\beqal
\tens H_{ij}(\vec x)&=\tens{\Gamma}_{i}^{0,\per}(\chi\vec e_j)(\vec x), \qquad\text{(symmetric if isotropy of $\tens \sigma_0$ ?)}
\eeqal 
and we derive this function with respect to $\overline{\vec{\tau}}$:
\begin{equation}
\frac{\partial \mathcal{E}}{\partial \overline{\vec{\tau}}}=-\int_Z{}^t\tens H\cdot\tens\sigma\cdot\left(\overline{\vec{E}}-\tens H\cdot\overline{\vec{\tau}}\right)
\end{equation}
which is null if
\begin{equation}
\int_Z{}^t\tens H\cdot\tens\sigma\cdot\overline{\vec{E}}=\int_Z{}^t\tens H\cdot\tens\sigma\cdot\tens H\cdot \overline{\vec{\tau}}.
\end{equation}
Because $\tens \sigma$ is positive definite we can (except if $\tens H$ is null everywhere) obtain a candidate $\overline{\vec{\tau}}$:
\begin{equation}
 \overline{\vec{\tau}}=\tens G^{-1}\int_Z{}^t\tens H\cdot\tens\sigma\cdot\overline{\vec{E}},
\end{equation}
where
\begin{equation}
\tens G=\int_Z{}^t\tens H\cdot\tens\sigma\cdot\tens H
\end{equation}
defines a positive definite quadratic form of $\tens{\Gamma}_0^\per$.
Coming back to the energy, we can state that
\begin{equation}
\begin{aligned}
\frac 12\overline{\vec E}\cdot\tens\sigma^\app\cdot\overline{\vec E}&\leq\mathcal{E}(\overline{\vec{\tau}})\\
&\leq\frac12\int_Z\overline{\vec{E}}\cdot\tens\sigma\cdot\overline{\vec{E}}-\frac 12\overline{\vec \tau}\cdot\tens G\cdot\overline{\vec \tau}\\
&\leq\frac 12\overline{\vec{E}}\cdot\tens\sigma^{Voigt}\cdot\overline{\vec{E}}-\frac 12\overline{\vec \tau}\cdot\tens G\cdot\overline{\vec \tau}\\
\end{aligned}
\end{equation}

\section{Minimisation of the complementary energy}
\subsection{Current density version of Lippmann-Schwinger equation}
It is known that another problem can be considered to compute another apparent conductivity:
\begin{gather}
  \label{eq1}
  Z:\quad\div{\vec{j}}=0,\qquad  \langle\vec j\rangle = \overline{\vec J},\\
  \label{eq3}
  Z:\quad\vec{j}=\tens{\sigma}\cdot\vec{E},\\
  \label{eq4}
  Z:\quad\vec{E}=\langle\vec E\rangle+\grad\phi^\per.
\end{gather}
where $\overline{\vec J}$ is a prescribed constant vector, and $\vec j$ remains a $Z$-antiperiodic field.

The above problem can be rewritten as
\begin{gather}
\label{eq11}
  Z:\quad\div{\vec{j}}=0, \qquad\langle\vec j\rangle = \overline{\vec J},\\
\label{eq22}
  Z:\quad\vec{E}=\tens{\rho}_0\cdot\vec{j}+\vec E^T,\qquad \vec E^T=(\tens\rho-\tens\rho_0)\cdot\vec j,\\
\label{eq33}
  Z:\quad\vec{E}=\langle\vec E\rangle+\grad\phi^\per\\
\end{gather}
where $\vec \rho=\vec \sigma^{-1}$.

And the solution of this problem is also the solution of this equation:
\begin{equation}
\label{eq_++}
\vec j= \overline{\vec J} + \tens\sigma_0\langle\vec E^T\rangle + \tens\sigma_0\cdot\tens\Gamma_0^\per\left(\tens\sigma_0\cdot\vec E^T\right)-\tens\sigma_0\cdot\vec E^T\qquad\text{with\ \ } \vec E^T=(\tens\rho-\tens\rho_0)\cdot\vec j.
\end{equation}
Indeed, (i) the average of this field is $\overline{\vec J}$, (ii) its divergence is null because the following equation is verified (by definition of $\tens\Gamma_0^\per$):
\begin{equation}
\div\left[\tens\sigma_0\cdot\tens\Gamma_0^\per\left(\tens\sigma_0\cdot\vec E^T\right)-\tens\sigma_0\cdot\vec E^T\right]=0
\end{equation}
and (iii) the corresponding field $\vec E=\tens{\rho}_0\cdot\vec{j}+\vec E^T$ can be written
\begin{equation}
\vec E=\tens{\rho}_0\cdot\vec{j}+\vec E^T=\tens{\rho}_0\cdot\overline{\vec{J}}+\langle\vec E^T\rangle+\tens\Gamma_0^\per\left(\tens\sigma_0\cdot\vec E^T\right)-\vec E^T+\vec E^T=\tens{\rho}_0\cdot\overline{\vec{J}}+\langle\vec E^T\rangle+\tens\Gamma_0^\per\left(\tens\sigma_0\cdot\vec E^T\right)
\end{equation}
which is Eq.\eqref{eq33} with $\langle\vec E\rangle=\tens{\rho}_0\cdot\overline{\vec{J}}+\langle\vec E^T\rangle$. Hence, rewriting Eq.\eqref{eq_++},
\begin{equation}
\vec j-\overline{\vec J} = -\tens\sigma_0\left(\vec E^T-\langle\vec E^T\rangle\right) + \tens\sigma_0\cdot\tens\Gamma_0^\per\left(\tens\sigma_0\cdot\vec E^T\right)\qquad\text{with\ \ } \vec E^T=(\tens\rho-\tens\rho_0)\cdot\vec j.
\end{equation}
Or
\begin{equation}
\vec j^* = \vec {\tau}^* - \tens\sigma_0\cdot\tens\Gamma_0^\per\left(\vec\tau\right), \qquad\text{where\ \ }\quad \vec \tau = -\tens\sigma_0\cdot\vec E^T = -\tens\sigma_0\cdot(\tens\rho-\tens\rho_0)\cdot\vec j
\end{equation}
and $\tens a^*$ is a notation for $\vec a-\langle\vec a\rangle$. This last equation is the current density version of Lippmann-Schwinger equation.

\subsection{Minimisation of the complementary energy}
It is well-known that solution of problem \eqref{eq1}-\eqref{eq4} is the solution of the minimisation problem:
\begin{equation}
\label{eq_EC}
  \underset{\vec j\in\mathcal S(\overline{\vec J})}{\min}\mathcal{W^{*}}(\vec j)
\end{equation}
where
\begin{equation}
\mathcal{W^{*}}(\vec j)=\frac 12\int_Z\vec j\cdot\tens\rho\cdot\vec j.
\end{equation}

We then take the polarization $\vec \tau = \chi \overline{\vec\tau}$ and the correponding current density:
\begin{equation}
\vec j=\overline{\vec J}+\vec \tau^*-\tens\sigma_0\cdot\tens\Gamma_0^\per\left(\vec\tau\right)=\overline{\vec J}+\chi^*\overline{\vec\tau}-\tens\sigma_0\cdot\tens\Gamma_0^\per\left(\chi\overline{\vec\tau}\right).
\end{equation}
This field is divergence-free, and its average is $\overline{\vec J}$.

We also define
\begin{equation}
\begin{aligned}
\mathcal{E^{*}}\left(\overline{\vec \tau}\right)&=\mathcal{W}^*\left(\overline{\vec J}+\chi^*\overline{\vec\tau}-\tens\sigma_0\cdot\tens\Gamma_0^\per\left(\chi\overline{\vec\tau}\right)\right)\\
&=\mathcal{W}^*\left(\overline{\vec J}-\tens F\cdot\overline{\vec\tau}\right)\\
&=\frac12\int_Z\left(\overline{\vec{J}}-\tens F\cdot\overline{\vec{\tau}}\right)\cdot\tens\rho\cdot\left(\overline{\vec{J}}-\tens F\cdot\overline{\vec{\tau}}\right)
\end{aligned}
\end{equation}
where
\begin{equation}
\tens F_{ij}(\vec x)=\left(\tens\sigma_0\cdot\tens{\Gamma}_0^{\per}\right)_{i}(\chi\vec e_j)(\vec x)-\delta_{ij}\,\chi^*(\vec x), \qquad\text{(which can be computed from $\tens H_{ij}$ !)}.
\end{equation} 
Minimising this function with respect to $\overline{\vec{\tau}}$:
\begin{equation}
\frac{\partial \mathcal{E^*}}{\partial \overline{\vec{\tau}}}=-\int_Z{}^t\tens F\cdot\tens\rho\cdot\left(\overline{\vec{J}}-\tens F\cdot\overline{\vec{\tau}}\right)
\end{equation}
which is null if
\begin{equation}
\int_Z{}^t\tens F\cdot\tens\rho\cdot\overline{\vec{J}}=\int_Z{}^t\tens F\cdot\tens\rho\cdot\tens F\cdot \overline{\vec{\tau}}.
\end{equation}
Because $\tens \rho$ is positive definite we can (except if $\tens F$ is null everywhere) obtain a candidate $\overline{\vec{\tau}}$:
\begin{equation}
 \overline{\vec{\tau}}=\tens K^{-1}\int_Z{}^t\tens F\cdot\tens\rho\cdot\overline{\vec{J}},
\end{equation}
where
\begin{equation}
\tens K=\int_Z{}^t\tens F\cdot\tens\rho\cdot\tens F.
\end{equation}
Coming back to the energy, we can state that
\begin{equation}
\begin{aligned}
\frac 12\overline{\vec J}\cdot\tens\rho^\app\cdot\overline{\vec J}&\leq\mathcal{E}^*(\overline{\vec{\tau}})\\
&\leq\frac12\int_Z\overline{\vec{J}}\cdot\tens\rho\cdot\overline{\vec{J}}-\frac 12\overline{\vec \tau}\cdot\tens K\cdot\overline{\vec \tau}\\
&\leq\frac 12\overline{\vec{J}}\cdot\tens\rho^{Reuss}\cdot\overline{\vec{J}}-\frac 12\overline{\vec \tau}\cdot\tens K\cdot\overline{\vec \tau}\\
\end{aligned}
\end{equation}
which also gives
\begin{equation}
\begin{aligned}
\frac 12\overline{\vec J}\cdot\tens\rho^\app\cdot\overline{\vec J}&\leq\frac 12\overline{\vec{J}}\cdot\tens\rho^{Reuss}\cdot\overline{\vec{J}}-\frac 12\overline{\vec J}\cdot\tens M\cdot\overline{\vec J}\\
\tens\sigma^\app&\geq\left[\tens\rho^{Reuss}-\tens M\right]^{-1}
\end{aligned}
\end{equation}
where
\begin{equation}
\tens M= \int_Z{}^t\tens\rho\cdot\tens F\cdot{}^t\tens K^{-1}\cdot\int_Z{}^t\tens F\cdot\tens\rho
\end{equation}


\section{Convergence of the upper bound}
We then consider the following electrical field:
\begin{equation}
\label{Efield}
\vec E(\overline{\vec{\tau}}_1,...,\overline{\vec{\tau}}_M)=\overline{\vec{E}}-\tens{\Gamma}_0^\per\left[\chi\overline{\vec{\tau}}_1-\chi\tens\delta\tens\sigma\cdot\tens{\Gamma}_0^\per\left(\chi\overline{\vec{\tau}}_2-\chi\tens\delta\tens\sigma\cdot\tens{\Gamma}_0^\per\left(...\right)\right)\right]=\overline{\vec{E}}-\sum_{k=1}^M\,\tens{\Gamma}_0^\per\left[\left(-\chi\tens\delta\tens\sigma\cdot\tens{\Gamma}_0^\per\right)^{k-1}\left(\chi\overline{\vec{\tau}}_k\right)\right]
\end{equation}
where \(\overline{\vec \tau}_k\) are constant vectors, and $\tens\delta\tens\sigma=\tens\sigma_1-\tens\sigma_0$.
We can write this field as a linear combination of the \(\overline{\vec \tau}_k\) :
\begin{equation}
\label{linearc}
\vec E(\overline{\vec{\tau}}_1,...,\overline{\vec{\tau}}_M)=\overline{\vec{E}}-\sum_{k=1}^M\,\tens H^k\cdot\overline{\vec{\tau}}_k
\end{equation}
where
\beqal
\tens H^k_{ij}(\vec x)&=\tens{\Gamma}_{i}^{0,\per}\left[\left(-\chi\tens\delta\tens\sigma\cdot\tens{\Gamma}_0^\per\right)^{k-1}(\chi\vec e_j)\right](\vec x)\\
&=\tens{\Gamma}_{i}^{0,\per}\left[-\chi\tens\delta\tens\sigma\cdot\tens H^{k-1}\right](\vec x)
\eeqal

We then define
\begin{equation}
\begin{aligned}
\mathcal{E}(\overline{\vec{\tau}}_1,...,\overline{\vec{\tau}}_M)&=\mathcal{W}\left(\vec E(\overline{\vec{\tau}}_1,...,\overline{\vec{\tau}}_M)\right)\\
&=\frac12\int_Z\left(\overline{\vec{E}}-\sum_{k=1}^M\,\tens H^k\cdot\overline{\vec{\tau}}_k\right)\cdot\tens\sigma\cdot\left(\overline{\vec{E}}-\sum_{k=1}^M\,\tens H^k\cdot\overline{\vec{\tau}}_k\right)
\end{aligned}
\end{equation}

and we derive this function with respect to $\overline{\vec{\tau}}_i$:
\begin{equation}
\frac{\partial \mathcal{E}}{\partial \overline{\vec{\tau}}_i}=-\int_Z{}^t\tens H^i\cdot\tens\sigma\cdot\left(\overline{\vec{E}}-\sum_{k=1}^M\,\tens H^k\cdot\overline{\vec{\tau}}_k\right)
\end{equation}
which is null if
\begin{equation}
\label{minimum}
\int_Z{}^t\tens H^i\cdot\tens\sigma\cdot\overline{\vec{E}}=\sum_{k=1}^{M}\int_Z{}^t\tens H^i\cdot\tens\sigma\cdot\tens H^k\cdot \overline{\vec{\tau}}_k.
\end{equation}
This is a linear matrix equation and we can write
\begin{equation}
\tens G_{ik}=\int_Z{}^t\tens H^i\cdot\tens\sigma\cdot\tens H^k\qquad\text{and\ \ \ }\tens U_i=\int_Z{}^t\tens H^i\cdot\tens\sigma\cdot\overline{\vec{E}}
\end{equation}

In fact these expressions can be reduced to integrals over the inclusions only.
We hence have
\beqal
\int_Z{}^t\tens H^i\cdot\tens\sigma_0\cdot\tens H^k&=\int_Z{}^t\tens H^i\cdot\tens\sigma_0\cdot\tens \Gamma_0\left(-\chi\delta\tens\sigma\cdot\tens H^{k-1}\right)\\
&=\int_Z{}^t\tens H^i\cdot\left[\tens\sigma_0\cdot\tens \Gamma_0\left(-\chi\delta\tens\sigma\cdot\tens H^{k-1}\right)+\chi\delta\tens\sigma\cdot\tens H^{k-1}-\chi\delta\tens\sigma\cdot\tens H^{k-1}\right]\\
&=-\int_Z\chi{}^t\tens H^i\cdot\delta\tens\sigma\cdot\tens H^{k-1}
\eeqal
where we have used the fact that $\tens\sigma_0\cdot\tens \Gamma_0\left(-\chi\delta\tens\sigma\cdot\tens H^{k-1}\right)+\chi\delta\tens\sigma\cdot\tens H^{k-1}$ is a statically admissible field.
We obtain
\beqal
\tens G_{ik}=\int_Z\chi{}^t\tens H^i\cdot\delta\tens\sigma\cdot\left[\tens H^k-\tens H^{k-1}\right]
\eeqal
and
\beqal
\tens G_{i1}=\int_Z\chi{}^t\tens H^i\cdot\left(\delta\tens\sigma\cdot\tens H^k+\tens 1\right)
\eeqal
And we also have
\beqal
\tens U_i=\int_Z\chi{}^t\tens H^i\cdot\delta\tens\sigma\cdot\overline{\vec{E}}
\eeqal
Coming back to the energy, we can state that
\begin{equation}
\begin{aligned}
\frac 12\overline{\vec E}\cdot\tens\sigma^\app\cdot\overline{\vec E}&\leq\mathcal{E}(\overline{\vec{\tau}}_1,...,\overline{\vec{\tau}}_M)\\
&\leq\frac12\int_Z\overline{\vec{E}}\cdot\tens\sigma\cdot\overline{\vec{E}}-\frac 12\overline{\vec \tau}_i\cdot\tens G_{ik}\cdot\overline{\vec \tau}_k\\
&\leq\frac 12\overline{\vec{E}}\cdot\tens\sigma^{Voigt}\cdot\overline{\vec{E}}-\frac 12\overline{\vec \tau}_i\cdot\tens G_{ik}\cdot\overline{\vec \tau}_k=\frac 12\overline{\vec{E}}\cdot\tens\sigma^{Voigt}\cdot\overline{\vec{E}}-\frac 12\overline{\vec \tau}_i\cdot\tens U_i\\
\end{aligned}
\end{equation}
(Einstein's convention on summation of indices)
which gives
\begin{equation}
\tens\sigma^\app \leq \tens\sigma^{Voigt} - \tens N
\end{equation}
with
\begin{equation}
\tens N= \int_Z{}^t\tens\sigma\cdot\tens H\cdot{}^t\tens G^{-1}\cdot\int_Z{}^t\tens H\cdot\tens\sigma
\end{equation}

\begin{remark}
Is there a convergence of the right-hand side member to the apparent conductivity when $M\to\infty$ ? I guess yes, because we know that the field obtained with the "basic scheme" is a special case of the field \eqref{Efield}, with all $\overline{\vec\tau}_k$ equal to the initial polarization of the basic scheme.
\end{remark}

\subsection{Contrast expansions}

We assume in the following that the two phases are isotropic, so that we can define the contrast $c=\sigma_1/\sigma_0$. In a first time, we want to find an expansion that permits to compute the bounds easily when changing the contrast. To that extent, let us rewrite $\tens H_{ij}^k(\vec x)$ as
\begin{equation}
\label{Hijk}
\tens H^k_{ij}(\vec x)=(c-1)^{k-1}\tens{\Gamma}_{i}^{0,\per}\left[\left(-\chi\sigma_0\tens{\Gamma}_0^\per\right)^{k-1}(\chi\vec e_j)\right](\vec x)
\end{equation}
and hence rewrite Eq. \eqref{linearc} as
\begin{equation}
\vec E(\overline{\vec{\tau}}_1,...,\overline{\vec{\tau}}_M)=\overline{\vec{E}}-\sum_{k=1}^M\,\tens A^k\cdot\overline{\vec{\pi}}_k
\end{equation}
where
\begin{equation}
\overline{\vec{\pi}}_k=(c-1)^{k-1}\overline{\vec{\tau}}_k
\end{equation}
and
\begin{equation}
\tens A^k_{ij}(\vec x)=\tens{\Gamma}_{i}^{0,\per}\left[\left(-\chi\sigma_0\tens{\Gamma}_0^\per\right)^{k-1}(\chi\vec e_j)\right](\vec x)=\tens{\Gamma}_{i}^{0,\per}\left[-\chi\sigma_0\tens A_{\bullet j}^{k-1}\right](\vec x)\qquad (k\geq 2)
\end{equation}
and
\beqal
\tens A_{ij}^1(\vec x)=\tens{\Gamma}_{i}^{0,\per}(\chi\vec e_j)(\vec x)
\eeqal
which is independent of the contrast.

Minimising the energy w. r. t. $\overline{\vec{\pi}}_k$, Eq. \eqref{minimum} must be replaced by
\begin{equation}
\int_Z{}^t\tens A^i\cdot\tens\sigma\cdot\overline{\vec{E}}=\sum_{k=1}^{M}\int_Z{}^t\tens A^i\cdot\tens\sigma\cdot\tens A^k\cdot \overline{\vec{\pi}}_k
\end{equation}

but this equation depends on the contrast. Hence, we rewrite it
\begin{equation}
\left[\sigma_0\int_Z(1-\chi)\,{}^t\tens A^i+\sigma_1\int_Z\chi\,{}^t\tens A^i\right]\cdot\overline{\vec{E}}=\left[\sigma_0\sum_{k=1}^{M}\int_Z(1-\chi)\,{}^t\tens A^i\cdot\tens A^k+\sigma_1\sum_{k=1}^{M}\int_Z\chi\,{}^t\tens A^i\cdot\tens A^k\right]\cdot \overline{\vec{\pi}}_k
\end{equation}
which can be re-written:
\beqal
\left[(\sigma_1-\sigma_0)\int_Z\chi\,{}^t\tens A^i\right]\cdot\overline{\vec{E}}=\left[\sum_{k=1}^{M}\int_Z\chi\,{}^t\tens A^i\cdot\left((\sigma_1-\sigma_0)\tens A^k-\sigma_0\tens A^{k-1}\right)\right]\cdot \overline{\vec{\pi}}_k
\eeqal
or also
\begin{equation}
\label{sys_cont}
(c-1)\tens P_i=\sum_{k=1}^M\left((c-1)\tens R_{ik}-\tens S_{ik}\right)\cdot\overline{\vec{\pi}}_k
\end{equation}
where
\begin{equation}
\label{subblocks}
\begin{aligned}
&\tens P_i=\int_Z\chi\,{}^t\tens A^i\cdot\overline{\vec{E}}\\
&\tens R_{ik}=\int_Z\chi\,{}^t\tens A^i\cdot\tens A^k\qquad\tens S_{ik}=\tens R_{i,(k-1)}.
\end{aligned}
\end{equation}
The upper bound can also be rewritten
\begin{equation}
\frac 12\overline{\vec E}\cdot\tens\sigma^\app\cdot\overline{\vec E}\leq\frac 12\overline{\vec{E}}\cdot\tens\sigma^{Voigt}\cdot\overline{\vec{E}}-(c-1)\,\sigma_0\overline{\vec \pi}_{i}\cdot \tens P_i.
\end{equation}

We could rewrite this inequality as
\begin{equation}
\overline{\vec E}\cdot\tens\sigma^\app\cdot\overline{\vec E}\leq(1-f)\,\sigma_0\,\overline{\vec{E}}\cdot\overline{\vec{E}}+c\,f\,\sigma_0\,\overline{\vec{E}}\cdot\overline{\vec{E}}-\sigma_0(c-1)^2\,\,\tens P\cdot\left((c-1)\tens R-\tens S\right)^{-1}\cdot\tens P
\end{equation}
where $\tens P,\tens R,\tens S$ do not depend on contrast, and whose expressions are given by the matrix sub-blocks and sub-vectors given by \eqref{subblocks}.

In fact the $\tens P_i,\tens R_{ik},\tens S_{ik}$ can be expressed with the Hill tensors. Indeed, assuming that each $\tens A^k$ is uniform on the inclusion (which is obviously a big approximation),
\begin{equation}
\label{subblocks}
\begin{aligned}
&\tens P_i=f\,(-\sigma_0)^i\left(\tens {}^t\tens P^0\right)^i\frac{1}{-\sigma_0}\cdot\overline{\vec E}=-\frac{f}{\sigma_0}\,\left(\frac{f-1}{3}\right)^i\overline{\vec E}\\
&\tens R_{ik}=f\,(-\sigma_0)^{i+k}\left({}^t\tens P^0\right)^i\cdot\left(\tens P^0\right)^k\frac{1}{\sigma_0^2}=\frac{f}{\sigma_0^2}\,\left(\frac{f-1}{3}\right)^{i+k}\tens 1\\
&\tens S_{ik}=\frac{f}{\sigma_0^2}\,\left(\frac{f-1}{3}\right)^{i+k-1}\tens 1
\end{aligned}
\end{equation}

so that
\begin{equation}
-\frac{f}{\sigma_0}(c-1)\,\left(\frac{f-1}{3}\right)^i\overline{\vec E}=f\frac{1}{\sigma_0^2}\left(\frac{f-1}{3}\right)^{i+k}\frac{c+2-f(c-1)}{1-f}\overline{\vec{\pi}}_k
\end{equation}
which is
\begin{equation}
-\delta\sigma\,\overline{\vec E}=\sum_k\left(\frac{f-1}{3}\right)^{k}\frac{c+2-f(c-1)}{1-f}\overline{\vec{\pi}}_k
\end{equation}
There is no possibility to invert the system.
If we had assumed all $\overline{\vec{\pi}}_k$ equals, we would have
\begin{equation}
\overline{\vec{\pi}}=\delta\sigma\frac{4-f}{c+2-f(c-1)}\,\overline{\vec E}
\end{equation}
and the apparent conductivity is bounded by the equation
\beqal
\frac{\sigma^\app}{\sigma_1}\leq\frac{1-f}{c}+f-f\,(1-f)\frac{(c-1)^2}{c}\frac{1}{c+2-f(c-1)}
\eeqal


We can also rewrite the inverse matrix in the right hand-side (assuming its existence):
\begin{equation}
\left(\frac 1c\tens R+\tens S\right)^{-1}=\tens S^{-1}\cdot(\frac 1c\tens R\cdot\tens S^{-1}+\tens{Id})^{-1}=\tens S^{-1}\cdot\sum_{p=0}^{\infty}\frac{(-1)^p}{c^p}\left(\tens R\cdot\tens S^{-1}\right)^p
\end{equation}
Assuming the convergence of our bounds to the apparent conductivity, this last expression would provide us a series expansion (on the contrast) that would converge to the effective conductivity (when $M\to\infty$, and hence, increasing the size of the matrices and vectors $\tens P, \tens Q,\tens R,\tens S$).

\section{Convergence of the lower bound}
We have the same result for the lower bound. We consider the following electrical current density field:
\begin{equation}
\vec j(\overline{\vec{\tau}}_1,...,\overline{\vec{\tau}}_M)=\overline{\vec{J}}-\sum_{k=1}^M\left[\langle\tens \tau^k\rangle-\tens \tau^k+\tens\sigma_0\cdot\,\tens{\Gamma}_0^\per\left(\tens \tau^k\right)\right]
\end{equation}
where
\begin{equation}
\tens \tau^k(\vec x)=\left(-\chi\tens\delta\tens\sigma\cdot\tens{\Gamma}_0^\per\right)^{k-1}\left(\chi\overline{\vec{\tau}}_k\right)(\vec x)
\end{equation}
and \(\overline{\vec \tau}_k\) are constant vectors, and $\tens\delta\tens\sigma=\tens\sigma_1-\tens\sigma_0$. We can rewrite the current density
as

\begin{equation}
\vec j(\overline{\vec{\tau}}_1,...,\overline{\vec{\tau}}_M)=\overline{\vec{J}}-\sum_{k=1}^M\,\tens F^k\cdot\overline{\vec{\tau}}_k
\end{equation}
where
\begin{equation}
\tens F^k_{ij}(\vec x)=\langle\left[\left(-\chi\tens\delta\tens\sigma\cdot\tens{\Gamma}_0^\per\right)^{k-1}\right]_i(\chi\vec e_j)(\vec x)\rangle-\left[\left(-\chi\tens\delta\tens\sigma\cdot\tens{\Gamma}_0^\per\right)^{k-1}\right]_i(\chi\vec e_j)(\vec x)+\tens\sigma_0\cdot\tens H^k_{ij}(\vec x).
\end{equation} 
We then define
\begin{equation}
\begin{aligned}
\mathcal{E^*}(\overline{\vec{\tau}}_1,...,\overline{\vec{\tau}}_M)&=\mathcal{W^*}\left(\vec j(\overline{\vec{\tau}}_1,...,\overline{\vec{\tau}}_M)\right)\\
&=\frac12\int_Z\left(\overline{\vec{J}}-\sum_{k=1}^M\,\tens F^k\cdot\overline{\vec{\tau}}_k\right)\cdot\tens\rho\cdot\left(\overline{\vec{J}}-\sum_{k=1}^M\,\tens F^k\cdot\overline{\vec{\tau}}_k\right)
\end{aligned}
\end{equation}

and we derive this function with respect to $\overline{\vec{\tau}}_i$:
\begin{equation}
\frac{\partial \mathcal{E^*}}{\partial \overline{\vec{\tau}}_i}=-\int_Z{}^t\tens F^i\cdot\tens\rho\cdot\left(\overline{\vec{J}}-\sum_{k=1}^M\,\tens F^k\cdot\overline{\vec{\tau}}_k\right)
\end{equation}
which is null if
\begin{equation}
\int_Z{}^t\tens F^i\cdot\tens\rho\cdot\overline{\vec{J}}=\sum_{k=1}^{M}\int_Z{}^t\tens F^i\cdot\tens\rho\cdot\tens F^k\cdot \overline{\vec{\tau}}_k.
\end{equation}
This is a linear matrix equation.

Coming back to the energy, we can state that
\begin{equation}
\begin{aligned}
\frac 12\overline{\vec J}\cdot\tens\rho^\app\cdot\overline{\vec J}&\leq\mathcal{E^*}(\overline{\vec{\tau}}_1,...,\overline{\vec{\tau}}_M)\\
&\leq\frac12\int_Z\overline{\vec{J}}\cdot\tens\rho\cdot\overline{\vec{J}}-\frac 12\overline{\vec \tau}_i\cdot\tens K_{ik}\cdot\overline{\vec \tau}_k\\
&\leq\frac 12\overline{\vec{J}}\cdot\tens\rho^{Reuss}\cdot\overline{\vec{J}}-\frac 12\overline{\vec \tau}_i\cdot\tens K_{ik}\cdot\overline{\vec \tau}_k\\
\end{aligned}
\end{equation}

which gives
\begin{equation}
\tens\sigma^\app \geq \left[\tens\rho^{Reuss} - \tens M\right]^{-1}
\end{equation}
with
\begin{equation}
\tens M= \int_Z{}^t\tens\rho\cdot\tens F\cdot{}^t\tens K^{-1}\cdot\int_Z{}^t\tens F\cdot\tens\rho
\end{equation}

\section{Equality of the bounds}

\begin{equation}
\begin{aligned}
&\tens\sigma^{Voigt}-\tens N=\left[\tens\rho^{Reuss}-\tens M\right]^{-1}\\
&\tens\sigma^{Voigt}\cdot\tens\rho^{Reuss}-\tens N\cdot\tens\rho^{Reuss}-\tens\sigma^{Voigt}\cdot\tens M+\tens N\cdot\tens M=\tens I\\
&(fc+1-f)\left(\tens N+c\tens M\right)-c\tens N\cdot\tens M=(1-c)\,(1-f)(fc+1-f)\tens I\\
\end{aligned}
\end{equation}

%\section{Hashin-Shtrikman principle}
%\section{Choice of the reference medium}



\label{Sec_conc}


\appendix



\end{document}

% Local Variables:
% fill-column: 80
% End:
