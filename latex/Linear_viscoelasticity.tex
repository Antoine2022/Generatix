\documentclass[times, final, 3p]{article}
\newcommand{\mytitle}{A solution for homogenization of isotropic maxwellian viscoelastic media}
\newcommand{\myjournal}{}
\newcommand{\myauthors}{A. Martin}

\title{\mytitle}
%\journal{\myjournal}


\usepackage[fleqn]{amsmath}

\usepackage{amssymb}
\usepackage{amsthm}
\newtheorem{remark}{Remark}

\usepackage{bm}
\usepackage{tikz}

%\usepackage[breaklinks=true, colorlinks=true, linktocpage=true, pdftitle={\mytitle}, pdfauthor={\myauthors}, pdfsubject={Preprint submitted to \myjournal}, urlcolor=blue]{hyperref}


\usepackage{xcolor}

\newcommand{\app}{\text{app}}
\newcommand{\D}{\mathrm d}
\newcommand{\dbldot}{\mathbin{\mathord{:}}}
\let\div\undefined
\DeclareMathOperator{\div}{div}
\newcommand{\e}{\mathrm{e}}
\newcommand{\vs}{\mathrm{v}}
\newcommand{\tr}{\mathrm{tr}}
\DeclareMathOperator{\grad}{\mathbf{grad}}
\newcommand{\per}{\mathrm{per}}
\newcommand{\reals}{\mathbb R}
\DeclareMathOperator{\sym}{\mathbf{sym}}
\newcommand{\tens}[1]{\vec{#1}}
\renewcommand{\vec}[1]{\bm{\mathrm{#1}}}
\newcommand{\Eb}{\overline{\vec{E}}}
\newcommand{\beq}{\begin{equation}}
\newcommand{\eeq}{\end{equation}}
\newcommand{\beqal}{\begin{equation}\begin{aligned}}
\newcommand{\eeqal}{\end{aligned}\end{equation}}
\newcommand{\tsigma}{\tens{\sigma}}
\newcommand{\tepsilon}{\tens{\varepsilon}}
\newcommand{\dtsigma}{\dot{\tens{\sigma}}}
\newcommand{\dtepsilon}{\dot{\tens{\varepsilon}}}

\bibliographystyle{elsarticle-harv}

\begin{document}

\begin{center}\section*{\mytitle}\end{center}
\bigskip
\section{Position of the problem}
We consider
a periodic cell \(Z\) of a random heterogeneous
material composed of two phases $1$ and $2$. The behaviour
is given by
\beq
\dtepsilon=\dtepsilon^\e+\dtepsilon^\vs
\eeq
with
\beqal
&\dtepsilon^\e=\dfrac{1}{3\kappa^\e}\dot{\sigma}^m\tens 1+\dfrac{1}{2\mu^\e}\dtsigma^d\\
&\dtepsilon^\vs=\dfrac{1}{3\kappa^\vs}\sigma^m\tens 1+\dfrac{1}{2\mu^\vs}\tsigma^d
\eeqal
where $\sigma^m=\dfrac13 \tr \tsigma$ and $\tsigma^d=\tsigma-\sigma^m\tens 1$.

For a biphasic medium, we note $\kappa_1^\e$ and $\kappa_2^\e$,
$\mu_1^\e$ and $\mu_2^\e$, etc.

Hence,
\beqal
\label{behaviour}
&\dtepsilon_i^m=\dfrac{1}{3\kappa_i^\e}\dot{\sigma}_i^m+\dfrac{1}{3\kappa_i^\vs}\sigma_i^m\\
&\dtepsilon_i^d=\dfrac{1}{2\mu_i^\e}\dtsigma_i^d+\dfrac{1}{2\mu_i^\vs}\tsigma_i^d
\eeqal
where $i=1,2$. Let us introduce $\tau^m_i=\dfrac{\kappa_i^\vs}{\kappa_i^\e}$
and $\tau^d_i=\dfrac{\mu_i^\vs}{\mu_i^\e}$.

We consider the following time-dependent mechanical problem:
\beqal
t=0 :& \quad\tsigma(\vec x,t)=\tsigma_0(\vec x),\quad \tepsilon(\vec x,t)=\tepsilon_0(\vec x)\\
t\geq0 :&\quad \tepsilon(\vec x,t)=\overline{\vec E}(t)+\grad^s\vec \xi^\per(\vec x,t)\\
&\quad\dtepsilon(\vec x,t)=\tens S(\vec x)\dbldot\dtsigma(\vec x,t)+\tens M(\vec x)\dbldot\tsigma(\vec x,t)\\
&\quad\div\tsigma(\vec x,t)=0
\eeqal
where $\overline{\vec E}(t)$ is given. We hence have
\beq
\dot{\varepsilon}^m(\vec x,t)=\dot{\overline{E}^m}(t)+\dot{e}^m(\vec x,t)\\
\dtepsilon^d(\vec x,t)=\dot{\overline{\vec E}^d}(t)+\dot{\tens e}^d(\vec x,t)\\
\eeq
where $\dot{e}^m(\vec x,t)+\dot{\tens e}^d(\vec x,t)=\grad^s\dot{\vec \xi}^\per(\vec x,t)$. We also assume that $\overline{\vec E}(t)$
can be decomposed as follows:
\beqal
&\dot{\overline{E}^m}(t)=\sum_{k=0}^M \dot{E}^m_ke^{-t/ \tau_k}\\
&\dot{\overline{\vec E}^d}(t)=\sum_{k=0}^M \dot{\tens E}^d_ke^{-t/ \tau_k}\\
\eeqal
where $\tau_1>...>\tau_M$ are characteristic times.
We let the possibility that one of the $\dot{E}^m_k$ or $\dot{\tens E}^d_k$ is null but not at the same time $\dot{E}^m_k=0$ and $\dot{\tens E}^d_k=\tens 0$.
\begin{remark}
It is equivalent to the hypothesis that $\overline{\vec E}(t)$ can be decomposed as a finite Prony serie.
\end{remark}

\section{Decomposition proposed for the strain field}
We consider the following decomposition for the strain field:
\beqal
\label{decompo}
&\dot{\varepsilon}^m(\vec x ,t)=\sum_{k=0}^M s_k(\vec x)e^{-t/ \tau_k}\\
&\dtepsilon^d(\vec x ,t)=\sum_{k=0}^M \tens d_k(\vec x)e^{-t/ \tau_k}\\
\eeqal
where
\beqal
&s_k(\vec x)=\dot{E}^m_k+\dot{e}^m_k(\vec x)\\
&\tens d_k(\vec x)=\dot{\tens E}^d_k+\dot{\tens e}^d_k(\vec x)\\
\eeqal
and $\dot{e}^m_k(\vec x)+\dot{\tens e}^d_k(\vec x)=\grad^s\dot{\vec \xi}_k^\per(\vec x)$.
Putting \eqref{decompo} into \eqref{behaviour}, we obtain a first-order linear time-differential
equation with source terms. Hence, by linearity, the solution $\tsigma$ can be decomposed as
\beqal
&\sigma^m(\vec x ,t)=\sum_{k=0}^M \sigma^m_k(\vec x,t)\\
&\tsigma^d(\vec x ,t)=\sum_{k=0}^M \tsigma^d_k(\vec x,t)\\
\eeqal
where each term (indexed by $k$) of the sums is solution of one of the following differential equations: 
\beqal
&s_k(\vec x)e^{-t/ \tau_k}=\dfrac{1}{3\kappa^\e(\vec x)}\dot{\sigma}_k^m(\vec x,t)+\dfrac{1}{3\kappa^\vs(\vec x)}\sigma_k^m(\vec x,t)\\
&\tens d_k(\vec x)e^{-t/ \tau_k}=\dfrac{1}{2\mu^\e(\vec x)}\dtsigma^d_k(\vec x,t)+\dfrac{1}{2\mu^\vs(\vec x)}\tsigma_k^d(\vec x,t)
\eeqal
for $k=0,...,M$. Let us fix $k$ in $[|0,M|]$. Hence, we can state that the solutions $\sigma^m_k$ and $\tsigma^d_k$
are given by (classical resolution of this type of equation):
\beqal
&\sigma_k^m(\vec x,t)=\zeta_k(\vec x) e^{-t/ \tau^m(\vec x)}+\dfrac{3\kappa^\vs(\vec x)s_k(\vec x)}{1/\tau^m(\vec x)-1/\tau_k}e^{-t/\tau_k}\\
&\tsigma_k^d(\vec x,t)=\tens \delta_k(\vec x) e^{-t/ \tau^d(\vec x)}+\dfrac{2\mu^\vs(\vec x)\tens d_k(\vec x)}{1/\tau^d(\vec x)-1/\tau_k}e^{-t/\tau_k}\\
\eeqal
We can then introduce
\beq
\tilde{\kappa}_k(\vec x)=\dfrac{\kappa^\vs(\vec x)}{1/\tau^m(\vec x)-1/\tau_k}\qquad\tilde{\mu}_k(\vec x)=\dfrac{\mu^\vs(\vec x)}{1/\tau^d(\vec x)-1/\tau_k}
\eeq
and hence
\beqal
\label{solution}
&\sigma^m(\vec x ,t)=\zeta(\vec x) e^{-t/ \tau^m(\vec x)}+\sum_{k=0}^M 3\tilde{\kappa}_k(\vec x)s_k(\vec x)e^{-t/\tau_k}\\
&\tsigma^d(\vec x ,t)=\tens \delta(\vec x) e^{-t/ \tau^d(\vec x)}+\sum_{k=0}^M 2\tilde{\mu}_k(\vec x)\tens d_k(\vec x)e^{-t/\tau_k}\\
\eeqal
where $\zeta(\vec x)=\sum_k\zeta_k(\vec x)$ and $\tens \delta(\vec x)=\sum_k\tens \delta_k(\vec x)$.

\section{Case of a finite number of phases}
\subsection{Case of a biphasic}
We can show that the equilibrium equation leads to
\beqal
&\div\left(3\tilde{\kappa}_k(\vec x)s_k(\vec x)\tens 1+2\tilde{\mu}_k(\vec x)\tens d_k(\vec x)\right)=0\quad \forall k=0,...,M\\
\eeqal
where
\beqal
\tilde{\kappa}_k^1&=\dfrac{\kappa^\vs_1}{1/\tau^m_1-1/\tau_k}\qquad\tilde{\mu}_k^2=\dfrac{\mu^\vs_2}{1/\tau^d_2-1/\tau_k}\\
\tilde{\kappa}_k^2&=\dfrac{\kappa^\vs_2}{1/\tau^m_2-1/\tau_k}\qquad\tilde{\mu}_k^2=\dfrac{\mu^\vs_2}{1/\tau^d_2-1/\tau_k}
\eeqal
This is a linear elastic homogenization problem of a biphasic medium.

Moreover, by the mean of equations \eqref{solution}, $\tens \delta(\vec x)$ and $\zeta(\vec x)$ are obtained with the initial
condition on the stress field. They must also check the following equations which are a consequence of the fact that they are statically admissible, but also that their contribution to the stress field is vanishing with different characteristic times.
\beqal
&\grad\left(\chi_i(\vec x)\zeta(\vec x)\right)=0\quad (i=1,2)\\
&\div\left(\chi_i(\vec x)\tens \delta(\vec x)\right)=0\quad (i=1,2)\\
\eeqal

\section{Numerical application}


\appendix



\end{document}

% Local Variables:
% fill-column: 80
% End:
