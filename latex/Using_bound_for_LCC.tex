\documentclass[times, final, 3p]{elsarticle}
\newcommand{\mytitle}{}
\newcommand{\myjournal}{}
\newcommand{\myauthors}{A. Martin}

\title{\mytitle}
%\journal{\myjournal}


\usepackage[fleqn]{amsmath}

\usepackage{amssymb}
\usepackage{amsthm}
\newtheorem{remark}{Remark}

\usepackage{bm}
\usepackage{tikz}

%\usepackage[breaklinks=true, colorlinks=true, linktocpage=true, pdftitle={\mytitle}, pdfauthor={\myauthors}, pdfsubject={Preprint submitted to \myjournal}, urlcolor=blue]{hyperref}


\usepackage{xcolor}

\newcommand{\app}{\text{app}}
\newcommand{\D}{\mathrm d}
\newcommand{\dbldot}{\mathbin{\mathord{:}}}
\let\div\undefined
\DeclareMathOperator{\div}{div}
\newcommand{\eff}{\mathrm{eff}}
\newcommand{\inter}{\mathrm{int}}
\newcommand{\EIM}{\mathrm{EIM}}
\newcommand{\Esh}{\mathrm{Esh}}
\DeclareMathOperator{\grad}{\mathbf{grad}}
\newcommand{\I}{\mathrm i}
\newcommand{\integers}{\mathbb Z}
\newcommand{\jump}[1]{[\![#1]\!]}
\newcommand{\M}{\mathrm{m}}
\newcommand{\per}{\mathrm{per}}
\newcommand{\PV}{\operatorname*{PV}}
\newcommand{\reals}{\mathbb R}
\DeclareMathOperator{\sym}{\mathbf{sym}}
\newcommand{\tens}[1]{\vec{#1}}
\renewcommand{\vec}[1]{\bm{\mathrm{#1}}}
\newcommand{\rot}[1]{\vec{rot}\ \vec{#1}}
\newcommand{\taumoy}{\overline{\vec{\tau}}}
\newcommand{\Eb}{\overline{\vec{E}}}

\bibliographystyle{elsarticle-harv}

\begin{document}
\begin{frontmatter}
  %\author[adr:1]{A. Martin\corref{cor:1}}\ead{}
  %\cortext[cor:1]{Corresponding author}
  \begin{abstract}
    
  \end{abstract}
  \begin{keyword}
    Homogenization\sep
    Conductivity\sep
   \end{keyword}
\end{frontmatter}

\section{The linear comparison composite}

We consider
a periodic cell \(Z\) of a random heterogeneous
material. The conductivity at \(\vec x\in Z\) is \(\tens\sigma(\vec x)\)
(symmetric, positive definite, second-order tensor); \(\vec E(\vec x)\),
\(\phi(\vec x)\) and \(\vec j(\vec x)\) denote the electric field, the electric
potential and the volumic current, respectively,
at point \(\vec x\). The apparent conductivity of the cell \(Z\),
\(\tens\sigma^\app\), is found from the solution to
the following problem
\begin{gather}
  \label{eq:20211209153049}
  Z:\quad\div{\vec{j}}=0,\\
  \label{eq:20211209153053}
  Z_r:\quad\vec{j}=\tens{\sigma}\cdot\vec{E}+\vec P,\\
  \label{eq:20211209153103}
  Z:\quad\vec{E}=\overline{\vec E}+\grad\phi^\per,
\end{gather}
where $\phi^\per$ is a $Z$-periodic field, $\vec{j}$ is a $Z$-antiperiodic field, and $\overline{\vec E}$
is a prescribed constant vector. Eq.~\eqref{eq:20211209153103} ensures that the electric field is curl-free,
with $\overline{\vec E}=\langle\vec E\rangle$, where angle brackets
denote volume averages over the cell \(Z\).
We consider a biphasic cell with a matrix whose conductivity is $\tens \sigma_0$ and another phase whose
conductivity is $\tens \sigma_1$. The polarization $\vec P$ is a $Z$-antiperiodic field.

We introduce the linear Green operator \(\tens{\Gamma}_0^\per\),
associated with \emph{a given arbitrary} conductivity \(\tens\sigma_0\).

The electrical field solution of problem \eqref{eq:20211209153049}-\eqref{eq:20211209153103} is also the solution of the minimization problem, which also defines the effective energy:
\begin{equation}
\mathcal{W}^\eff\left(\overline{\vec E}\right)=\underset{\vec E\in\mathcal C(\overline{\vec E})}{\min}\mathcal{W}(\vec E)
\end{equation}
where
\begin{equation}
\label{eq_EP}
  \mathcal{W}(\vec E)=\int_Z\frac12\vec E\cdot\tens\sigma\cdot\vec E+\vec P\cdot\vec E.
\end{equation}

The Lippmann-Schwinger equation is 
\begin{equation}
\vec E = \overline{\vec E} - \tens\Gamma_0^\per\left(\vec\tau\right), \qquad\text{where\ \ }\quad \vec \tau =\left(\tens\sigma -\tens\sigma_0\right)\cdot\vec E+\vec P
\end{equation}

We then consider the following electrical field:
\begin{equation}
\vec E(\overline{\vec{\tau}}_1,...,\overline{\vec{\tau}}_M)=\overline{\vec{E}}-\tens{\Gamma}_0^\per(\vec P)-\sum_{k=1}^M\,\tens H^k\cdot\overline{\vec{\tau}}_k
\end{equation}
where \(\overline{\vec \tau}_k\) are constant vectors and $\chi$ denotes characteristic function of medium whose conductivity is $\tens \sigma_1$.
This field is curl-free, and its average is $\overline{\vec{E}}$. We then define
\begin{equation}
\begin{aligned}
\mathcal{E}(\overline{\vec{\tau}}_1,...,\overline{\vec{\tau}}_M)&=\mathcal{W}\left(\vec E(\overline{\vec{\tau}}_1,...,\overline{\vec{\tau}}_M)\right)\\
&=\int_Z\left(\overline{\vec{E}}-\tens{\Gamma}_0^\per(\vec P)-\sum_{k=1}^M\,\tens H^k\cdot\overline{\vec{\tau}}_k\right)\cdot\left\{\frac{\tens\sigma}{2}\cdot\left(\overline{\vec{E}}-\tens{\Gamma}_0^\per(\vec P)-\sum_{k=1}^M\,\tens H^k\cdot\overline{\vec{\tau}}_k\right)+\vec P\right\}
\end{aligned}
\end{equation}

and we derive this function with respect to $\overline{\vec{\tau}}_i$:
\begin{equation}
\label{stat_cond}
\frac{\partial \mathcal{E}}{\partial \overline{\vec{\tau}}_i}=-\int_Z{}^t\tens H^i\cdot\left\{\tens\sigma\cdot\left(\overline{\vec{E}}-\tens{\Gamma}_0^\per(\vec P)-\sum_{k=1}^M\,\tens H^k\cdot\overline{\vec{\tau}}_k\right)+\vec P\right\}
\end{equation}
which is null if
\begin{equation}
\int_Z{}^t\tens H^i\cdot\left\{\tens\sigma\cdot\left(\overline{\vec{E}}-\tens{\Gamma}_0^\per(\vec P)\right)+\vec P\right\}=\sum_{k=1}^{M}\int_Z{}^t\tens H^i\cdot\tens\sigma\cdot\tens H^k\cdot \overline{\vec{\tau}}_k.
\end{equation}
This is a linear matrix equation and we can define
\begin{equation}
\tens G_{ik}=\int_Z{}^t\tens H^i\cdot\tens\sigma\cdot\tens H^k\qquad\text{and\ \ \ }\tens U_i=\int_Z{}^t\tens H^i\cdot\left\{\tens\sigma\cdot\left(\overline{\vec{E}}-\tens{\Gamma}_0^\per(\vec P)\right)+\vec P\right\}=\tens a_i\cdot\overline{\vec E}+\tens b_i
\end{equation}
where
...

Coming back to the energy, we write
\begin{equation}
\begin{aligned}
\mathcal{E}(\overline{\vec{\tau}}_1,...,\overline{\vec{\tau}}_M)&=\int_Z\left[\left(\overline{\vec{E}}-\tens{\Gamma}_0^\per(\vec P)-\sum_{k=1}^M\,\tens H^k\cdot\overline{\vec{\tau}}_k\right)\cdot\left\{\frac{\tens\sigma}{2}\cdot\left(\overline{\vec{E}}-\tens{\Gamma}_0^\per(\vec P)-\sum_{k=1}^M\,\tens H^k\cdot\overline{\vec{\tau}}_k\right)+\frac{\vec P}{2}+\frac{\vec P}{2}\right\}\right]\\
&=\int_Z\left[\left(\overline{\vec{E}}-\tens{\Gamma}_0^\per(\vec P)\right)\cdot\left\{\frac{\tens\sigma}{2}\cdot\left(\overline{\vec{E}}-\tens{\Gamma}_0^\per(\vec P)-\sum_{k=1}^M\,\tens H^k\cdot\overline{\vec{\tau}}_k\right)+\frac{\vec P}{2}\right\}-\sum_{k=1}^M\,\frac{\vec P}{2}\cdot\tens H^k\cdot\overline{\vec{\tau}}_k\right]\\
&=\int_Z\left[\left(\overline{\vec{E}}-\tens{\Gamma}_0^\per(\vec P)\right)\cdot\left(\frac{\tens\sigma}{2}\cdot\left(\overline{\vec{E}}-\tens{\Gamma}_0^\per(\vec P)\right)+\frac{\vec P}{2}\right)-\sum_{k=1}^M\,\left(\left(\overline{\vec{E}}-\tens{\Gamma}_0^\per(\vec P)\right)\cdot\frac{\tens\sigma}{2}+\frac{\vec P}{2}\right)\cdot\tens H^k\cdot\overline{\vec{\tau}}_k\right]\\
&=\left\{\int_Z\left(\overline{\vec{E}}-\tens{\Gamma}_0^\per(\vec P)\right)\cdot\left(\frac{\tens\sigma}{2}\cdot\left(\overline{\vec{E}}-\tens{\Gamma}_0^\per(\vec P)\right)+\frac{\vec P}{2}\right)\right\}-\frac12\sum_{k,i=1}^M\,\overline{\vec{\tau}}_i\cdot\tens U_i\\
\end{aligned}
\end{equation}
where we used \eqref{stat_cond} from line 1 to line 2, and also definition of $\tens U_i$ from line 3 to line 4.
Hence, we have
\begin{equation}
\begin{aligned}
\mathcal{E}(\overline{\vec{\tau}}_1,...,\overline{\vec{\tau}}_M)&=\frac12\overline{\vec{E}}\cdot\tens\sigma^{Voigt}\cdot\overline{\vec{E}}+\left\{\int_Z-\tens{\Gamma}_0^\per(\vec P)\cdot\tens\sigma\cdot\overline{\vec{E}}+\frac12\tens{\Gamma}_0^\per(\vec P)\cdot\tens\sigma\cdot\tens{\Gamma}_0^\per(\vec P)+\left(\overline{\vec{E}}-\tens{\Gamma}_0^\per(\vec P)\right)\cdot\frac{\vec P}{2}\right\}-\frac12\sum_{k,i=1}^M\,\overline{\vec{\tau}}_i\cdot\tens U_i\\
\end{aligned}
\end{equation}
and we have also
\begin{equation}
\sum_{k,i=1}^M\,\overline{\vec{\tau}}_i\cdot\tens U_i=\left(\overline{\vec E}\cdot{}^t\tens a+{}^t\tens b\right)\cdot\tens G^{-1}\cdot\left(\tens a\cdot\overline{\vec E}+\tens b\right)=\overline{\vec E}\cdot{}^t\tens a\cdot\tens G^{-1}\cdot\tens a\cdot\overline{\vec E}+\left({}^t\tens b\cdot\tens G^{-1}\cdot\tens a+{}^t\tens a\cdot\tens G^{-1}\cdot\tens b\right)\cdot\overline{\vec E}+{}^t\tens b\cdot\tens G^{-1}\cdot\tens b
\end{equation}
And then 
\begin{equation}
\label{inequality}
\mathcal{W}^\eff\left(\overline{\vec E}\right)\leq\frac12\overline{\vec E}\cdot\tens\sigma^\eff\cdot\overline{\vec E}+\vec P^\eff\cdot\overline{\vec E}+w^0
\end{equation}
where
\begin{equation}
\begin{aligned}
\tens\sigma^\eff&=\tens\sigma^{Voigt}-{}^t\tens a\cdot\tens G^{-1}\cdot\tens a\qquad\tens P^\eff=\int_Z\left(\frac{\vec P}{2}-\tens\sigma\cdot\tens{\Gamma}_0^\per(\vec P)\right)-\frac12\left({}^t\tens b\cdot\tens G^{-1}\cdot\tens a+{}^t\tens a\cdot\tens G^{-1}\cdot\tens b\right)\\
w^0&=\frac12\int_Z\left[\tens{\Gamma}_0^\per(\vec P)\cdot\tens\sigma\cdot\tens{\Gamma}_0^\per(\vec P)-\tens{\Gamma}_0^\per(\vec P)\cdot\vec P\right]-\frac12{}^t\tens b\cdot\tens G^{-1}\cdot\tens b
\end{aligned}
\end{equation}
In case of convergence, the inequality \eqref{inequality} becomes an equality.

\section{If $\vec P$ is a constant-piecewise field}

\section{Minimisation of the complementary energy}
\subsection{Current density version of Lippmann-Schwinger equation}
It is known that another problem can be considered to compute another apparent conductivity:
\begin{gather}
  \label{eq1}
  Z:\quad\div{\vec{j}}=0,\qquad  \langle\vec j\rangle = \overline{\vec J},\\
  \label{eq3}
  Z:\quad\vec{j}=\tens{\sigma}\cdot\vec{E},\\
  \label{eq4}
  Z:\quad\vec{E}=\langle\vec E\rangle+\grad\phi^\per.
\end{gather}
where $\overline{\vec J}$ is a prescribed constant vector, and $\vec j$ remains a $Z$-antiperiodic field.

The above problem can be rewritten as
\begin{gather}
\label{eq11}
  Z:\quad\div{\vec{j}}=0, \qquad\langle\vec j\rangle = \overline{\vec J},\\
\label{eq22}
  Z:\quad\vec{E}=\tens{\rho}_0\cdot\vec{j}+\vec E^T,\qquad \vec E^T=(\tens\rho-\tens\rho_0)\cdot\vec j,\\
\label{eq33}
  Z:\quad\vec{E}=\langle\vec E\rangle+\grad\phi^\per\\
\end{gather}
where $\vec \rho=\vec \sigma^{-1}$.

And the solution of this problem is also the solution of this equation:
\begin{equation}
\label{eq_++}
\vec j= \overline{\vec J} + \tens\sigma_0\langle\vec E^T\rangle + \tens\sigma_0\cdot\tens\Gamma_0^\per\left(\tens\sigma_0\cdot\vec E^T\right)-\tens\sigma_0\cdot\vec E^T\qquad\text{with\ \ } \vec E^T=(\tens\rho-\tens\rho_0)\cdot\vec j.
\end{equation}
Indeed, (i) the average of this field is $\overline{\vec J}$, (ii) its divergence is null because the following equation is verified (by definition of $\tens\Gamma_0^\per$):
\begin{equation}
\div\left[\tens\sigma_0\cdot\tens\Gamma_0^\per\left(\tens\sigma_0\cdot\vec E^T\right)-\tens\sigma_0\cdot\vec E^T\right]=0
\end{equation}
and (iii) the corresponding field $\vec E=\tens{\rho}_0\cdot\vec{j}+\vec E^T$ can be written
\begin{equation}
\vec E=\tens{\rho}_0\cdot\vec{j}+\vec E^T=\tens{\rho}_0\cdot\overline{\vec{J}}+\langle\vec E^T\rangle+\tens\Gamma_0^\per\left(\tens\sigma_0\cdot\vec E^T\right)-\vec E^T+\vec E^T=\tens{\rho}_0\cdot\overline{\vec{J}}+\langle\vec E^T\rangle+\tens\Gamma_0^\per\left(\tens\sigma_0\cdot\vec E^T\right)
\end{equation}
which is Eq.\eqref{eq33} with $\langle\vec E\rangle=\tens{\rho}_0\cdot\overline{\vec{J}}+\langle\vec E^T\rangle$. Hence, rewriting Eq.\eqref{eq_++},
\begin{equation}
\vec j-\overline{\vec J} = -\tens\sigma_0\left(\vec E^T-\langle\vec E^T\rangle\right) + \tens\sigma_0\cdot\tens\Gamma_0^\per\left(\tens\sigma_0\cdot\vec E^T\right)\qquad\text{with\ \ } \vec E^T=(\tens\rho-\tens\rho_0)\cdot\vec j.
\end{equation}
Or
\begin{equation}
\vec j^* = \vec {\tau}^* - \tens\sigma_0\cdot\tens\Gamma_0^\per\left(\vec\tau\right), \qquad\text{where\ \ }\quad \vec \tau = -\tens\sigma_0\cdot\vec E^T = -\tens\sigma_0\cdot(\tens\rho-\tens\rho_0)\cdot\vec j
\end{equation}
and $\tens a^*$ is a notation for $\vec a-\langle\vec a\rangle$. This last equation is the current density version of Lippmann-Schwinger equation.

\subsection{Minimisation of the complementary energy}
It is well-known that solution of problem \eqref{eq1}-\eqref{eq4} is the solution of the minimisation problem:
\begin{equation}
\label{eq_EC}
  \underset{\vec j\in\mathcal S(\overline{\vec J})}{\min}\mathcal{W^{*}}(\vec j)
\end{equation}
where
\begin{equation}
\mathcal{W^{*}}(\vec j)=\frac 12\int_Z\vec j\cdot\tens\rho\cdot\vec j.
\end{equation}

We then take the polarization $\vec \tau = \chi \overline{\vec\tau}$ and the correponding current density:
\begin{equation}
\vec j=\overline{\vec J}+\vec \tau^*-\tens\sigma_0\cdot\tens\Gamma_0^\per\left(\vec\tau\right)=\overline{\vec J}+\chi^*\overline{\vec\tau}-\tens\sigma_0\cdot\tens\Gamma_0^\per\left(\chi\overline{\vec\tau}\right).
\end{equation}
This field is divergence-free, and its average is $\overline{\vec J}$.

We also define
\begin{equation}
\begin{aligned}
\mathcal{E^{*}}\left(\overline{\vec \tau}\right)&=\mathcal{W}^*\left(\overline{\vec J}+\chi^*\overline{\vec\tau}-\tens\sigma_0\cdot\tens\Gamma_0^\per\left(\chi\overline{\vec\tau}\right)\right)\\
&=\mathcal{W}^*\left(\overline{\vec J}-\tens F\cdot\overline{\vec\tau}\right)\\
&=\frac12\int_Z\left(\overline{\vec{J}}-\tens F\cdot\overline{\vec{\tau}}\right)\cdot\tens\rho\cdot\left(\overline{\vec{J}}-\tens F\cdot\overline{\vec{\tau}}\right)
\end{aligned}
\end{equation}
where
\begin{equation}
\tens F_{ij}(\vec x)=\left(\tens\sigma_0\cdot\tens{\Gamma}_0^{\per}\right)_{i}(\chi\vec e_j)(\vec x)-\delta_{ij}\,\chi^*(\vec x), \qquad\text{(which can be computed from $\tens H_{ij}$ !)}.
\end{equation} 
Minimising this function with respect to $\overline{\vec{\tau}}$:
\begin{equation}
\frac{\partial \mathcal{E^*}}{\partial \overline{\vec{\tau}}}=-\int_Z{}^t\tens F\cdot\tens\rho\cdot\left(\overline{\vec{J}}-\tens F\cdot\overline{\vec{\tau}}\right)
\end{equation}
which is null if
\begin{equation}
\int_Z{}^t\tens F\cdot\tens\rho\cdot\overline{\vec{J}}=\int_Z{}^t\tens F\cdot\tens\rho\cdot\tens F\cdot \overline{\vec{\tau}}.
\end{equation}
Because $\tens \rho$ is positive definite we can (except if $\tens F$ is null everywhere) obtain a candidate $\overline{\vec{\tau}}$:
\begin{equation}
 \overline{\vec{\tau}}=\tens K^{-1}\int_Z{}^t\tens F\cdot\tens\rho\cdot\overline{\vec{J}},
\end{equation}
where
\begin{equation}
\tens K=\int_Z{}^t\tens F\cdot\tens\rho\cdot\tens F.
\end{equation}
Coming back to the energy, we can state that
\begin{equation}
\begin{aligned}
\frac 12\overline{\vec J}\cdot\tens\rho^\app\cdot\overline{\vec J}&\leq\mathcal{E}^*(\overline{\vec{\tau}})\\
&\leq\frac12\int_Z\overline{\vec{J}}\cdot\tens\rho\cdot\overline{\vec{J}}-\frac 12\overline{\vec \tau}\cdot\tens K\cdot\overline{\vec \tau}\\
&\leq\frac 12\overline{\vec{J}}\cdot\tens\rho^{Reuss}\cdot\overline{\vec{J}}-\frac 12\overline{\vec \tau}\cdot\tens K\cdot\overline{\vec \tau}\\
\end{aligned}
\end{equation}
which also gives
\begin{equation}
\begin{aligned}
\frac 12\overline{\vec J}\cdot\tens\rho^\app\cdot\overline{\vec J}&\leq\frac 12\overline{\vec{J}}\cdot\tens\rho^{Reuss}\cdot\overline{\vec{J}}-\frac 12\overline{\vec J}\cdot\tens M\cdot\overline{\vec J}\\
\tens\sigma^\app&\geq\left[\tens\rho^{Reuss}-\tens M\right]^{-1}
\end{aligned}
\end{equation}
where
\begin{equation}
\tens M= \int_Z{}^t\tens\rho\cdot\tens F\cdot{}^t\tens K^{-1}\cdot\int_Z{}^t\tens F\cdot\tens\rho
\end{equation}


\section{Convergence of the upper bound}
We then consider the following electrical field:
\begin{equation}
\vec E(\overline{\vec{\tau}}_1,...,\overline{\vec{\tau}}_M)=\overline{\vec{E}}-\sum_{k=1}^M\,\tens{\Gamma}_0^\per\left[\left(-\chi\tens\delta\tens\sigma\cdot\tens{\Gamma}_0^\per\right)^{k-1}\left(\chi\overline{\vec{\tau}}_k\right)\right]
\end{equation}
where \(\overline{\vec \tau}_k\) are constant vectors, and $\tens\delta\tens\sigma=\tens\sigma_1-\tens\sigma_0$.
We can write this field as a linear combination of the \(\overline{\vec \tau}_k\) :
\begin{equation}
\vec E(\overline{\vec{\tau}}_1,...,\overline{\vec{\tau}}_M)=\overline{\vec{E}}-\sum_{k=1}^M\,\tens H^k\cdot\overline{\vec{\tau}}_k
\end{equation}
where
\begin{equation}
\tens H^k_{ij}(\vec x)=\tens{\Gamma}_{i}^{0,\per}\left[\left(-\chi\tens\delta\tens\sigma\cdot\tens{\Gamma}_0^\per\right)^{k-1}(\chi\vec e_j)\right](\vec x), \qquad\text{(symmetric if isotropy of $\tens \sigma_0$ ?)}.
\end{equation} 
We then define
\begin{equation}
\begin{aligned}
\mathcal{E}(\overline{\vec{\tau}}_1,...,\overline{\vec{\tau}}_M)&=\mathcal{W}\left(\vec E(\overline{\vec{\tau}}_1,...,\overline{\vec{\tau}}_M)\right)\\
&=\frac12\int_Z\left(\overline{\vec{E}}-\sum_{k=1}^M\,\tens H^k\cdot\overline{\vec{\tau}}_k\right)\cdot\tens\sigma\cdot\left(\overline{\vec{E}}-\sum_{k=1}^M\,\tens H^k\cdot\overline{\vec{\tau}}_k\right)
\end{aligned}
\end{equation}

and we derive this function with respect to $\overline{\vec{\tau}}_i$:
\begin{equation}
\frac{\partial \mathcal{E}}{\partial \overline{\vec{\tau}}_i}=-\int_Z{}^t\tens H^i\cdot\tens\sigma\cdot\left(\overline{\vec{E}}-\sum_{k=1}^M\,\tens H^k\cdot\overline{\vec{\tau}}_k\right)
\end{equation}
which is null if
\begin{equation}
\int_Z{}^t\tens H^i\cdot\tens\sigma\cdot\overline{\vec{E}}=\sum_{k=1}^{M}\int_Z{}^t\tens H^i\cdot\tens\sigma\cdot\tens H^k\cdot \overline{\vec{\tau}}_k.
\end{equation}
This is a linear matrix equation.

Coming back to the energy, we can state that
\begin{equation}
\begin{aligned}
\frac 12\overline{\vec E}\cdot\tens\sigma^\app\cdot\overline{\vec E}&\leq\mathcal{E}(\overline{\vec{\tau}}_1,...,\overline{\vec{\tau}}_M)\\
&\leq\frac12\int_Z\overline{\vec{E}}\cdot\tens\sigma\cdot\overline{\vec{E}}-\frac 12\overline{\vec \tau}\cdot\tens G\cdot\overline{\vec \tau}\\
&\leq\frac 12\overline{\vec{E}}\cdot\tens\sigma^{Voigt}\cdot\overline{\vec{E}}-\frac 12\overline{\vec \tau}\cdot\tens G\cdot\overline{\vec \tau}\\
\end{aligned}
\end{equation}

\subsection{$n^{th}$ order bound on the contrast}

We can show that previous expansion provides the $n^{th}$ order bound on the contrast.

\section{Convergence of the lower bound}
We then consider the following electrical current density field:
\begin{equation}
\vec j(\overline{\vec{\tau}}_1,...,\overline{\vec{\tau}}_M)=\overline{\vec{J}}-\sum_{k=1}^M\left[\langle\tens \tau^k\rangle-\tens \tau^k+\tens\sigma_0\cdot\,\tens{\Gamma}_0^\per\left(\tens \tau^k\right)\right]
\end{equation}
where
\begin{equation}
\tens \tau^k(\vec x)=\left(-\chi\tens\delta\tens\sigma\cdot\tens{\Gamma}_0^\per\right)^{k-1}\left(\chi\overline{\vec{\tau}}_k\right)(\vec x)
\end{equation}
and \(\overline{\vec \tau}_k\) are constant vectors, and $\tens\delta\tens\sigma=\tens\sigma_1-\tens\sigma_0$. We can rewrite the current density
as

\begin{equation}
\vec j(\overline{\vec{\tau}}_1,...,\overline{\vec{\tau}}_M)=\overline{\vec{J}}-\sum_{k=1}^M\,\tens F^k\cdot\overline{\vec{\tau}}_k
\end{equation}
where
\begin{equation}
\tens F^k_{ij}(\vec x)=\langle\left[\left(-\chi\tens\delta\tens\sigma\cdot\tens{\Gamma}_0^\per\right)^{k-1}\right]_i(\chi\vec e_j)(\vec x)\rangle-\left[\left(-\chi\tens\delta\tens\sigma\cdot\tens{\Gamma}_0^\per\right)^{k-1}\right]_i(\chi\vec e_j)(\vec x)+\tens\sigma_0\cdot\tens H^k_{ij}(\vec x).
\end{equation} 
We then define
\begin{equation}
\begin{aligned}
\mathcal{E^*}(\overline{\vec{\tau}}_1,...,\overline{\vec{\tau}}_M)&=\mathcal{W^*}\left(\vec j(\overline{\vec{\tau}}_1,...,\overline{\vec{\tau}}_M)\right)\\
&=\frac12\int_Z\left(\overline{\vec{J}}-\sum_{k=1}^M\,\tens F^k\cdot\overline{\vec{\tau}}_k\right)\cdot\tens\rho\cdot\left(\overline{\vec{J}}-\sum_{k=1}^M\,\tens F^k\cdot\overline{\vec{\tau}}_k\right)
\end{aligned}
\end{equation}

and we derive this function with respect to $\overline{\vec{\tau}}_i$:
\begin{equation}
\frac{\partial \mathcal{E^*}}{\partial \overline{\vec{\tau}}_i}=-\int_Z{}^t\tens F^i\cdot\tens\rho\cdot\left(\overline{\vec{J}}-\sum_{k=1}^M\,\tens F^k\cdot\overline{\vec{\tau}}_k\right)
\end{equation}
which is null if
\begin{equation}
\int_Z{}^t\tens F^i\cdot\tens\rho\cdot\overline{\vec{J}}=\sum_{k=1}^{M}\int_Z{}^t\tens F^i\cdot\tens\rho\cdot\tens F^k\cdot \overline{\vec{\tau}}_k.
\end{equation}
This is a linear matrix equation.

Coming back to the energy, we can state that
\begin{equation}
\begin{aligned}
\frac 12\overline{\vec J}\cdot\tens\rho^\app\cdot\overline{\vec J}&\leq\mathcal{E^*}(\overline{\vec{\tau}}_1,...,\overline{\vec{\tau}}_M)\\
&\leq\frac12\int_Z\overline{\vec{J}}\cdot\tens\rho\cdot\overline{\vec{J}}-\frac 12\overline{\vec \tau}\cdot\tens K\cdot\overline{\vec \tau}\\
&\leq\frac 12\overline{\vec{J}}\cdot\tens\rho^{Reuss}\cdot\overline{\vec{J}}-\frac 12\overline{\vec \tau}\cdot\tens K\cdot\overline{\vec \tau}\\
\end{aligned}
\end{equation}


\section{Hashin-Shtrikman principle}
\section{Choice of the reference medium}



\label{Sec_conc}


\appendix



\end{document}

% Local Variables:
% fill-column: 80
% End:
