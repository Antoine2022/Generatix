\documentclass[times, final, 3p]{article}
\newcommand{\mytitle}{}
\newcommand{\myjournal}{}
\newcommand{\myauthors}{A. Martin}

\title{\mytitle}
%\journal{\myjournal}


\usepackage[fleqn]{amsmath}

\usepackage{amssymb}
\usepackage{amsthm}
\newtheorem{remark}{Remark}

\usepackage{bm}
\usepackage{tikz}

%\usepackage[breaklinks=true, colorlinks=true, linktocpage=true, pdftitle={\mytitle}, pdfauthor={\myauthors}, pdfsubject={Preprint submitted to \myjournal}, urlcolor=blue]{hyperref}


\usepackage{xcolor}

\newcommand{\app}{\text{app}}
\newcommand{\D}{\mathrm d}
\newcommand{\dbldot}{\mathbin{\mathord{:}}}
\let\div\undefined
\DeclareMathOperator{\div}{div}
\newcommand{\eff}{\mathrm{eff}}
\newcommand{\inter}{\mathrm{int}}
\newcommand{\EIM}{\mathrm{EIM}}
\newcommand{\Esh}{\mathrm{Esh}}
\DeclareMathOperator{\grad}{\mathbf{grad}}
\newcommand{\I}{\mathrm i}
\newcommand{\integers}{\mathbb Z}
\newcommand{\jump}[1]{[\![#1]\!]}
\newcommand{\M}{\mathrm{m}}
\newcommand{\per}{\mathrm{per}}
\newcommand{\PV}{\operatorname*{PV}}
\newcommand{\reals}{\mathbb R}
\DeclareMathOperator{\sym}{\mathbf{sym}}
\newcommand{\tens}[1]{\vec{#1}}
\renewcommand{\vec}[1]{\bm{\mathrm{#1}}}
\newcommand{\rot}[1]{\vec{rot}\ \vec{#1}}
\newcommand{\taumoy}{\overline{\vec{\tau}}}
\newcommand{\Eb}{\overline{\vec{E}}}

\bibliographystyle{elsarticle-harv}

\begin{document}


\section{Continuity equation in a periodic electrodynamic medium}
We consider
a periodic cell \(Z\) of a random heterogeneous
material. The conductivity at \(\vec x\in Z\) is \(\tens\sigma(\vec x)\)
(symmetric, positive definite, second-order tensor); \(\vec E(\vec x)\),
\(\phi(\vec x)\) and \(\vec j(\vec x)\) denote the electric field, the electric
potential and the volumic current, respectively,
at point \(\vec x\). The permittivity is noted $\tens \epsilon(\vec x)$.
\begin{gather}
  \label{eq:20211209153049}
  Z:\quad\div\left(\vec{j}+\frac{\D}{\D t}\vec D\right)=0,\\
  \label{eq:20211209153053}
  Z:\quad\vec{j}=\tens{\sigma}\cdot\vec{E},\\
  \label{eq_D}
  Z:\quad\vec{D}=\tens{\epsilon}\cdot\vec{E},\\
  \label{eq:20211209153103}
  Z:\quad\vec{E}=\overline{\vec E}+\grad\phi^\per,
\end{gather}
where $\phi^\per$ is a $Z$-periodic field, $\vec{j}$ is a $Z$-antiperiodic field, and $\overline{\vec E}$
is a prescribed constant vector. Eq.~\eqref{eq:20211209153103} ensures that the electric field is curl-free,
with $\overline{\vec E}=\langle\vec E\rangle$, where angle brackets
denote volume averages over the cell \(Z\).


\section{The time dependence}
We consider the following electric field as a data:
\begin{equation}
\Eb=t\Eb_1+\frac{t^2}{2}\Eb_2+\frac{t^3}{6}\Eb_3
\end{equation}
and we assume the following development for $\phi^\per$
\begin{equation}
\phi^\per=\sum_{i=1}^{\infty}\phi_i \frac{t^i}{i!}
\end{equation}
and we note $\grad\phi^\per=\vec e$, and also $\grad\phi_i=\vec e_i$ (for all $i$).
Hence, we have the following problems:
\begin{equation}
\begin{aligned}
&\div\left(\tens \epsilon\cdot(\Eb_1+\vec e_1)\right)=0\\
&\div\left(\tens \sigma\cdot(\Eb_1+\vec e_1)+\tens \epsilon\cdot(\Eb_2+\vec e_2)\right)=0\\
&\div\left(\tens \sigma\cdot(\Eb_2+\vec e_2)+\tens \epsilon\cdot(\Eb_3+\vec e_3)\right)=0\\
&\div\left(\tens \sigma\cdot(\Eb_3+\vec e_3)+\tens \epsilon\cdot\vec e_4\right)=0\\
&\div\left(\tens \sigma\cdot\vec e_4+\tens \epsilon\cdot\vec e_5\right)=0\\
&\div\left(\tens \sigma\cdot\vec e_5+\tens \epsilon\cdot\vec e_6\right)=0\\
...
\end{aligned}
\end{equation}
These problems can be resolved in an iterative way.

\section{Resolution of the problems}
\subsection{first problem}
We consider a biphasic cell with a matrix whose conductivity is isotropic $\sigma_0$ and another phase whose
conductivity is isotropic $\sigma_1$. The permittivity is also isotropic, $\epsilon_0=\tau_0\sigma_0$ for medium 0
and $\epsilon_1=\tau_1\sigma_1$ for medium 1. Hence, the first problem to solve is
\begin{equation}
\div\left(\epsilon(\Eb_1+\vec e_1)\right)=0\\
\end{equation}
and its solution can be expressed as
\begin{equation}
\Eb_1+\vec e_1=\Eb_1-\tens A(\vec x)\cdot\tens G^{-1}\cdot\vec U^1
\end{equation}
where the subblocks of $\tens G$ are:
\begin{equation}
\tens G_{ik}=\int_Z\epsilon\,{}^t\tens A^i\cdot\tens A^k
\end{equation}
and
\begin{equation}
\tens U^1_i=\int_Z\epsilon\,{}^t\tens A^i\cdot\Eb_1\qquad\text{and}\quad\tens A^{k}_{ij}(\vec x)=\tens{\Gamma}_{i}^{0,\per}\left[\left(-\chi\epsilon_0\tens{\Gamma}_0^\per\right)^{k-1}(\chi\vec e_j)\right](\vec x)
\end{equation} 

\subsection{second problem}
The second problem to solve is
\begin{equation}
\div\left(\epsilon(\Eb_2+\vec e_2)+\vec P_1\right)=0\\
\end{equation}
with $\vec P_1=\sigma(\Eb_1+\vec e_1)=\sigma\left(\Eb_1-\tens A(\vec x)\cdot\tens G^{-1}\cdot\vec U^1\right)$. Its solution can be expressed as
\begin{equation}
\Eb_2+\vec e_2=\Eb_2-\tens A(\vec x)\cdot\tens G^{-1}\cdot\vec U^2-\tens{\Gamma}_0^\per(\vec P_1)
\end{equation}
where
\begin{equation}
\tens U^2_i=\int_Z{}^t\tens A^i\cdot\left[\epsilon\left(\Eb_2-\tens{\Gamma}_0^\per(\vec P_1)\right)+\vec P_1\right]
\end{equation}

\subsection{third problem}
The third problem to solve is
\begin{equation}
\div\left(\epsilon(\Eb_3+\vec e_3)+\vec P_2\right)=0\\
\end{equation}
with $\vec P_2=\sigma(\Eb_2+\vec e_2)=\sigma\left(\Eb_2-\tens A(\vec x)\cdot\tens G^{-1}\cdot\vec U^2-\tens{\Gamma}_0^\per(\vec P_1)\right)$.
Its solution can be expressed as
\begin{equation}
\Eb_3+\vec e_3=\Eb_3-\tens A(\vec x)\cdot\tens G^{-1}\cdot\vec U^3-\tens{\Gamma}_0^\per(\vec P_2)
\end{equation}
where
\begin{equation}
\tens U^3_i=\int_Z{}^t\tens A^i\cdot\left[\epsilon\left(\Eb_3-\tens{\Gamma}_0^\per(\vec P_2)\right)+\vec P_2\right]
\end{equation}

\subsection{fourth problem}
The fourth problem to solve is
\begin{equation}
\div\left(\epsilon\,\vec e_4+\vec P_3\right)=0\\
\end{equation}
with $\vec P_3=\sigma(\Eb_3+\vec e_3)=\sigma\left(\Eb_3-\tens A(\vec x)\cdot\tens G^{-1}\cdot\vec U^3-\tens{\Gamma}_0^\per(\vec P_2)\right)$.
Its solution can be expressed as
\begin{equation}
\vec e_4=-\tens A(\vec x)\cdot\tens G^{-1}\cdot\vec U^4-\tens{\Gamma}_0^\per(\vec P_3)
\end{equation}
where
\begin{equation}
\tens U^4_i=\int_Z{}^t\tens A^i\cdot\left[-\epsilon\,\tens{\Gamma}_0^\per(\vec P_3)+\vec P_3\right]
\end{equation} 

\subsection{$k^{th}$ problem with $k\geq 5$}
The $k^{th}$ problem to solve is
\begin{equation}
\div\left(\epsilon\,\vec e_k+\vec P_{k-1}\right)=0\\
\end{equation}
with $\vec P_{k-1}=\sigma\vec e_{k-1}=\sigma\left(-\tens A(\vec x)\cdot\tens G^{-1}\cdot\vec U^{k-1}-\tens{\Gamma}_0^\per(\vec P_{k-2})\right)$.
Its solution can be expressed as
\begin{equation}
\vec e_k=-\tens A(\vec x)\cdot\tens G^{-1}\cdot\vec U^k-\tens{\Gamma}_0^\per(\vec P_{k-1})
\end{equation}
where
\begin{equation}
\tens U^k_i=\int_Z{}^t\tens A^i\cdot\left[-\epsilon\,\tens{\Gamma}_0^\per(\vec P_{k-1})+\vec P_{k-1}\right]
\end{equation} 

\section{Numerical application}
We choose $\sigma_0=1$, $\tau_0=1e-6$, so that $\epsilon_0=1e-6$; and
$\sigma_1=1e6$, $\tau_1=1e-6$, so that $\epsilon_1=1$.
For the remote field, we choose $\Eb_1=4e5$, $\Eb_2=-4e5$, $\Eb_3=0$ and hence the time of interest
is between $1e-6$ and $1e-5$.



\label{Sec_conc}


\appendix



\end{document}

% Local Variables:
% fill-column: 80
% End:
