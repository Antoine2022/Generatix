\documentclass[times, final, 3p]{article}
\newcommand{\mytitle}{}
\newcommand{\myjournal}{}
\newcommand{\myauthors}{A. Martin}

\title{\mytitle}
%\journal{\myjournal}


\usepackage[fleqn]{amsmath}

\usepackage{amssymb}
\usepackage{amsthm}
\newtheorem{remark}{Remark}

\usepackage{bm}
\usepackage{tikz}

%\usepackage[breaklinks=true, colorlinks=true, linktocpage=true, pdftitle={\mytitle}, pdfauthor={\myauthors}, pdfsubject={Preprint submitted to \myjournal}, urlcolor=blue]{hyperref}


\usepackage{xcolor}

\newcommand{\app}{\text{app}}
\newcommand{\D}{\mathrm d}
\newcommand{\dbldot}{\mathbin{\mathord{:}}}
\let\div\undefined
\DeclareMathOperator{\div}{div}
\newcommand{\eff}{\mathrm{eff}}
\newcommand{\inter}{\mathrm{int}}
\newcommand{\EIM}{\mathrm{EIM}}
\newcommand{\Esh}{\mathrm{Esh}}
\DeclareMathOperator{\grad}{\mathbf{grad}}
\newcommand{\I}{\mathrm i}
\newcommand{\integers}{\mathbb Z}
\newcommand{\jump}[1]{[\![#1]\!]}
\newcommand{\M}{\mathrm{m}}
\newcommand{\per}{\mathrm{per}}
\newcommand{\PV}{\operatorname*{PV}}
\newcommand{\reals}{\mathbb R}
\DeclareMathOperator{\sym}{\mathbf{sym}}
\newcommand{\tens}[1]{\vec{#1}}
\renewcommand{\vec}[1]{\bm{\mathrm{#1}}}
\newcommand{\rot}[1]{\vec{rot}\ \vec{#1}}
\newcommand{\taumoy}{\overline{\vec{\tau}}}
\newcommand{\Eb}{\overline{\vec{E}}}

\bibliographystyle{elsarticle-harv}

\begin{document}


\section{Produit scalaire}
Soit $\tens G_1(\vec x)$ et $\tens G(\vec x)$ deux champs vectoriels
tridimensionnels (donc 3 composantes chacun), on aimerait réaliser l'intégrale
suivante :
\begin{equation}
\int_{\vec x_\alpha\in M_\alpha}\tens G_1(\vec x_\alpha)\cdot\tens G_2(\vec x_\alpha)\,\D^3\,\vec x_\alpha
\end{equation}
où $M_\alpha$ est le matériau $\alpha$.
Par ailleurs, soit un champ vectoriel 3d $\tens G(\vec x)$, on aimerait pouvoir construire un champ (vectoriel 3d) $\tens G_2(\vec x)$
tel que 
\begin{equation}
\forall \vec x,\quad\tens G(\vec x)=f(\tens G_2(\vec x))
\end{equation}
où $f$ est une fonction simple (cette fonctionnalité existe peut-être déjà ?)




\end{document}

% Local Variables:
% fill-column: 80
% End:
