\documentclass[times, final, 3p]{article}
\newcommand{\mytitle}{}
\newcommand{\myjournal}{}
\newcommand{\myauthors}{A. Martin}

\title{\mytitle}
%\journal{\myjournal}


\usepackage[fleqn]{amsmath}

\usepackage{amssymb}
\usepackage{amsthm}
\newtheorem{remark}{Remark}

\usepackage{bm}
\usepackage{tikz}

%\usepackage[breaklinks=true, colorlinks=true, linktocpage=true, pdftitle={\mytitle}, pdfauthor={\myauthors}, pdfsubject={Preprint submitted to \myjournal}, urlcolor=blue]{hyperref}


\usepackage{xcolor}

\newcommand{\app}{\text{app}}
\newcommand{\D}{\mathrm d}
\newcommand{\dbldot}{\mathbin{\mathord{:}}}
\let\div\undefined
\DeclareMathOperator{\div}{div}
\newcommand{\eff}{\mathrm{eff}}
\DeclareMathOperator{\grad}{\mathbf{grad}}
\newcommand{\I}{\mathrm i}
\newcommand{\integers}{\mathbb Z}
\newcommand{\jump}[1]{[\![#1]\!]}
\newcommand{\M}{\mathrm{m}}
\newcommand{\per}{\mathrm{per}}
\newcommand{\beqal}{\begin{equation}\begin{aligned}}
\newcommand{\eeqal}{\end{aligned}\end{equation}}
\newcommand{\PV}{\operatorname*{PV}}
\newcommand{\reals}{\mathbb R}
\DeclareMathOperator{\sym}{\mathbf{sym}}
\newcommand{\tens}[1]{\vec{#1}}
\renewcommand{\vec}[1]{\bm{\mathrm{#1}}}
\newcommand{\rot}[1]{\vec{rot}\ \vec{#1}}
\newcommand{\taumoy}{\overline{\vec{\tau}}}
\newcommand{\Eb}{\overline{\vec{E}}}

\bibliographystyle{elsarticle-harv}

\begin{document}


\section{Minimisation of Hashin-Shtrikman functional}

We consider
a periodic cell \(Z\) of a random heterogeneous
material. The conductivity at \(\vec x\in Z\) is \(\tens\sigma(\vec x)\)
(symmetric, positive definite, second-order tensor); \(\vec E(\vec x)\),
\(\phi(\vec x)\) and \(\vec j(\vec x)\) denote the electric field, the electric
potential and the volumic current, respectively,
at point \(\vec x\). The apparent conductivity of the cell \(Z\),
\(\tens\sigma^\app\), is found from the solution to
the following problem
\begin{gather}
  \label{eq:20211209153049}
  Z:\quad\div{\vec{j}}=0,\\
  \label{eq:20211209153053}
  Z:\quad\vec{j}=\tens{\sigma}\cdot\vec{E},\\
  \label{eq:20211209153103}
  Z:\quad\vec{E}=\overline{\vec E}+\grad\phi^\per,
\end{gather}
where $\phi^\per$ is a $Z$-periodic field, $\vec{j}$ is a $Z$-antiperiodic field, and $\overline{\vec E}$
is a prescribed constant vector. Eq.~\eqref{eq:20211209153103} ensures that the electric field is curl-free,
with $\overline{\vec E}=\langle\vec E\rangle$, where angle brackets
denote volume averages over the cell \(Z\).
We consider a biphasic cell with a matrix whose conductivity is $\tens \sigma_0$ and another phase whose
conductivity is $\tens \sigma_1$.

We introduce the linear Green operator \(\tens{\Gamma}_0^\per\),
associated with the conductivity \(\tens\sigma_0\).

The electrical field solution of problem \eqref{eq:20211209153049}-\eqref{eq:20211209153103} minimizes the Hashin-Shtrikman functional
\begin{equation}
\label{HS}
  \mathcal{H}(\vec \tau)=\frac 12\overline{\vec E}\cdot\tens \sigma_0\cdot\overline{\vec E}-\frac 12\langle\vec \tau\cdot \tens \rho\cdot\vec\tau\rangle-\frac 12 \langle\vec\tau\cdot\tens \Gamma(\vec\tau)\rangle+\overline{\vec E}\cdot\langle\vec\tau\rangle
\end{equation}
with $\tens \rho = \tens\delta\tens\sigma^{-1}$.

We then consider the following polarization field:
\begin{equation}
\vec \tau=\sum_{k=0}^N\left(-\chi\tens\delta\tens\sigma\tens{\Gamma}_0^\per\right)^k(\chi\overline{\vec{\tau}}_k)\equiv\sum_{k=0}^N\vec{\tau}_k(\vec x)
\end{equation}
For $N=0$, we obtain the following polarization field:
\begin{equation}
\vec \tau_0=\chi\overline{\vec{\tau}}_0
\end{equation}
and for $N=1$, the following one:
\begin{equation}
\vec \tau_1=\chi\overline{\vec{\tau}}_0-\chi\tens\delta\tens\sigma\tens{\Gamma}_0^\per(\chi\overline{\vec{\tau}}_1)
\end{equation}

The different terms of the Hashin-Shtrikman functional hence are:
\beqal
-\frac 12\langle\vec \tau\cdot \tens \rho\cdot\vec\tau\rangle=-\frac 12\sum_k\sum_l\overline{\vec{\tau}}_k\cdot\langle{}^t\tens A^k\dbldot\tens\rho\dbldot\tens A^l\rangle\cdot\overline{\vec \tau}_l
\eeqal
with
\beqal
\left[\tens A^k\right]_{ij} (\vec x)=\left[\left(-\chi\tens\delta\tens\sigma\tens{\Gamma}_0^\per\right)^k\right]_i(\chi\tens e_j)(\vec x)
\eeqal





\appendix



\end{document}

% Local Variables:
% fill-column: 80
% End:
